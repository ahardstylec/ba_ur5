\chapter{Konzept}
\label{konzept_kon}

\section{Datenbank}
\label{datenbank_kon}

Für die Realisierung von \sona{euSDB} ist der Einsatz von zwei Datenbanksystemen
geplant, eines für die gesamte
Datenverwaltung (Kapitel \ref{datenverwaltung_i_kon}), das andere
für die schnelle Abwicklung von Suchanfragen an das System (Kapitel
\ref{datenverwaltung_ii_kon}). Dies bietet mehrere Vorteile: Eine angemessene
Datenverwaltung, die eine übersichtliche Datenstruktur bietet und mit Hilfe von
\ac{SQL}-Befehlen, auch ohne Nutzung des \sona{Django}-Frameworks, gepflegt und
administriert werden kann und eine Suchdatenbank, die die modernen Anforderungen
nach Datenintegrität, Skalierbarkeit und Erweiterbarkeit in vollem Umfang
abdeckt.

\subsection{Datenbank zur Verwaltung von Sicherheitsdatenblättern}
\label{datenverwaltung_i_kon}

Für die Datenverwaltung und Speicherung kommt ein \sona{MySQL}-Datenbanksystem
zum Einsatz. \sona{MySQL} wird von \fina{Oracle} weiterentwickelt und vermarktet.
Es gibt eine kostenlose und eine kostenpflichtige
Enterpriseversion. \cite{ORACOR-2013}
\\
Neben einer Nutzer- und Rechteverwaltung, wird in der Datenbank auch eine
Logtabelle geführt. Diese speichert jede Änderung am Datenbestand. Die
Kernaufgabe besteht in der Verwaltung der \ac{SDB}.
\begin{figure}
  \centering
    \includegraphics[width=0.99\textwidth]{pic/db_entwurf.png}
      \caption[Datenbankstruktur der Plattform euSDB]{Die Graphik zeigt die
      Datenbankstruktur (Entwurf) von euSDB.
      Aus Gründen der Übersichtlichkeit zeigen die Tabellen nur wenige Felder. Im Anhang Din A3/2}
      \label{fig:db_entwurf}
\end{figure}
\\
Ein Sicherheitsdatenblatt wird durch die in Kapitel \ref{fachliche_domaene}
vorgestellten Kennzeichnungen (siehe Tabelle \ref{tab:meinetabelle}) und
Hinweise beschrieben. Diese Kennzeichnungen sind für alle Chemikalien einheitlich und
erfahren in absehbarer Zeit keine Änderungen. Die H-, P-, R- und S-Sätze,
GHS-, EU- und Hazard-Gefahrenkennungen werden mehrfach
verwendet und liegen in jeweils einer eigenen Tabelle in der Datenbank
vor (siehe Graphik \ref{fig:db_entwurf}).
\\
Wird eine Anweisung oder Gefahrenkennung mit einer Chemikalie verknüpft,
geschieht dies über eine Matchingtabelle. Eine Solche Matchingtabelle enthält
dann einen Fremdschlüssel (Foreign-Key) der Chemikalie und eine mit der
Chemikalie verknüpfte Anweisung bzw. Gefahreneinstufung. Bei der Verknüpfung
von Chemikalien und Anweisungen bzw. Gefahreneinstufung handelt es sich um
sogenannte Many-To-Many Beziehungen.
\\
Many-To-Many Beziehungen bedeuten in diesem Fall, dass eine Anweisung mit
mehreren Chemikalien verknüpft und eine Chemikalie mit mehreren
Anweisungen verknüpft sein kann.
\\
Bei der herstellenden Firma verhält es sich anders. Ein \ac{SDB}
wird immer nur von einer Firma für die von ihr hergestellte Chemikalie
herausgegeben. Ein \ac{SDB} einer Chemikalie hat genau einen Hersteller. In
diesem Fall ist es möglich direkt den Fremdschlüssel in der Tabelle der
Chemikalie abzulegen. \ac{SDB} gleicher Chemikalien
verschiedener Hersteller sind nicht identisch.
\\
Jeder Hersteller gibt leicht unterschiedliche Daten zu Chemikalien an. So
ist es theoretisch möglich, dass sich die Angaben zum Siedepunkt unterscheiden.
Aus diesem Grund werden daher gleiche Chemikalien unterschiedlicher Hersteller
als unterschiedliche Chemikalien betrachtet. Auch geben die Hersteller gerne
unterschiedliche Einsatz- und Verwendungszwecke an. Zur Veranschaulichung ein
Beispiel: Ein Farbenhersteller stellt selbst Aceton her. Es wird zur Verdünnung
von Farben eingesetzt. Verkauft der Farbenhersteller sein Aceton, wird er im
\ac{SDB} angeben, dass es zur Verdünnung von Farben zu verwenden
ist. Ein Reinigungsmittelhersteller wird vermutlich in seinem
\ac{SDB} angeben, dass es sich um ein Reinigungsmittel handelt und
dafür einzusetzen ist. Dies macht es ebenfalls sinnvoll, gleiche Chemikalien
verschiedener Hersteller als unterschiedliche Chemikalien zu speichern.
\\
Die Datenbank ist in ihrer Struktur sehr einfach gehalten. Das macht das
Migrieren und Warten der Daten sehr leicht. Der Einsatz eines
Standarddatenbanksystems gewährleistet einen langfristigen Support auf vielen
Plattformen. Ein ständiger Umzug von einem Datenbanksystem zu einem anderen ist
bei Plattformwechseln nicht nötig.
\\
Eine Chemikalie selbst wird dabei durch ungefähr 60 Parameter
(Siedetemperatur, Dichte, molare Masse, \ldots) gekennzeichnet. Hinzukommen noch
die Angaben zur Einstufung der Gefahren und sich aus den Gefahren ergebende
Sicherheits- und Arbeitsanweisungen. Jeder dieser Parameter stellt ein eigenes
Feld in der Tabelle der Chemikalien dar.
\\
Hinzu kommt noch die Möglichkeit einer Versionierung. Häufig verändern sich im
Laufe der Zeit die Angaben auf einem \ac{SDB}. Jede Veränderung
geht mit der Herausgabe einer neuen Version eines \ac{SDB}
einher. Um die Informationen, die in einem \ac{SDB} aufgeführt sind,
lange nachhalten zu können, existiert nicht nur für die Chemikalie selbst eine
Versionierung, sondern auch für alle mit ihr verknüpften Komponenten. Sollte
sich nun eine Einstufung oder ein Satz ändern, kann man trotzdem nachschlagen,
wie ein früheres \ac{SDB} aufgebaut war.

\subsection{Datenbank zur Realisierung einer effektiven Volltextsuche}
\label{datenverwaltung_ii_kon}

\begin{figure}[h]
  \centering
    \includegraphics[width=0.8\textwidth]{pic/aufbau_es_dat_chem.jpg}
      \caption[Aufbau von elasticsearch]{Entwurf des Aufbaus von elasticsearch.
      Es gibt einen Node mit fünf Shards. Der Index ChemicalIndex verteilt sich
      auf diesen Node und die Shards. Beispielhaft sind hier drei Dokumente auf
      Shard 2 und 4 verteilt worden}
      \label{fig:aufbau_es_dat_chem}
\end{figure}
\\
Das System, das zur Suche eingesetzt wird, heißt \sona{elasticsearch}. Das
System ist zunächst auf einen Node mit fünf Shards begrenzt. Für jede Chemikalie
wird in dem Index "`ChemicalIndex"' ein Dokument abgelegt. Dieses enthält die
möglichen Identifizierer. Diese sollen nach internationalen Vorgaben möglichst
eindeutige Bezeichner (z.B. Name der Chemikalie, internationale Formeln,
internationale Identifizierungsnummern) sein. Nur nach solchen zu suchen ist sinnvoll.
Wird ein neuer Datensatz einer Chemikalie erstellt oder der einer bestehenden
Chemikalie bearbeitet, sollen die Änderungen automatisch auch auf die in
\sona{elasticsearch} abgelegten Daten übertragen werden. Auf diese Weise hält
sich der Verwaltungsaufwand, die Suchfunktion betreffend, im laufenden Betrieb
in Grenzen.
\\
Später ist es möglich den Suchindex noch um Einträge zu erweitern oder
neue Suchindizes zu erstellen. Auch kann das gesamte Suchsystem zu einem
verteilten System umgebaut werden.

\section{Aufbau des Djangoprojekts}
\label{aufbau_djangoprojekt_kon}

Die Grundstruktur eines Djangoprojektes ist im Kapitel \ref{sec:django_gru}
Grundlagen beschrieben. Das System setzt sich aus den folgenden
Grundfunktionen zusammen: (1) Beschreibung von Chemikalien mit ihren \ac{SDB},
(2) die Suche nach Chemikalien und \ac{SDB} und (3) eine Benutzer- und
Rechteverwaltung (siehe Abbildung \ref{fig:django_module}).
\begin{figure}[h]
  \centering
    \includegraphics[width=0.8\textwidth]{pic/django_module.jpg}
      \caption[Aufbau des euSDB-Django-Projektes]{Das Projekt euSDB besteht aus
      den drei Kernmodulen chemicals, search und userauth.}
      \label{fig:django_module}
\end{figure}
\\
(1) Soll dabei dem Kunden nicht nur die chemischen Daten und \ac{SDB} zur
Verfügung stellen, sondern regelt die Versionierung und langfristige Speicherung
von \ac{SDB}. Dazu nutzt diese Funktion eine \ac{REST}-konforme \ac{API}, die
nicht nur den menschlichen manuellen, sondern auch den automatisierten Zugriff
durch Programme und Maschinen erlaubt. Diese Funktion vermag es, \ac{SDB} in den
Formaten \ac{JSON}, \ac{XML}, \ac{HTML}, \ac{PDF} auszugeben.
\\
(2) \sona{elasticserach} ermöglicht die gezielte Suche nach einzelnen \ac{SDB}.
Nicht immer führen die angegebenen Werte, nach denen gesucht wird, zu eindeutigen Ergebnissen. Aus
diesem Grund wird immer eine Liste mit den möglichen \ac{SDB} zurückgegeben.
Diese kann kein Element, ein Element oder mehrere Elemente enthalten.
\\
(3) Die Administrationsoberfläche erlaubt es dem Administrator, Rechte an die
Benutzer zu vergeben, Benutzer in Gruppen zusammenzufassen und diesen Gruppen
Rechte zuzuweisen, damit auf die \ac{SDB} zugegriffen werden kann.
\\
Durch die hohe Funktionalität von \sona{Python} und den Aufbau des Projektes mit
Hilfe des \sona{Django}-Frameworks wird eine langfristige Nutzung der Plattform
angestrebt.

\section{Aufbau der REST-konformen Schnittstelle}
\label{aufbau_rest_schnittstelle_kon}

Die \ac{REST}-konforme \ac{API} soll vier Ausgabeformate haben: \ac{JSON}, \ac{XML},
\ac{HTML} und \ac{PDF} (siehe Abbildung \ref{fig:ausgabeformate}). Dabei
soll das Ausgabeformat sowohl im \ac{HTTP}-Header, als auch als Parameter an die
\ac{URI} angehängt werden können. 
\begin{lstlisting}[
    caption={Beispielanfragen, bei denen das Ausgabeformat jeweils header- und
    parametergesteuert ist},
    label=lst:bsp_abfrage_kon,
    captionpos=b
]
https://www.euSDB.de/chemicals/?format=json 
https://www.euSDB.de/chemicals/?format=xml
curl -X GET -H "Accept: application/json" https://www.euSDB.de/chemicals/
curl -X GET -H "Accept: application/xml" https://www.euSDB.de/chemicals/
\end{lstlisting}
\\
Auf diese Weise soll nach der Eingabe von Zeile 1 des Listing
\ref{lst:bsp_abfrage_kon} im Browser ein \ac{JSON}-Dokument angezeigt
werden, nach Eingabe von Zeile 2 des Listing \ref{lst:bsp_abfrage_kon} ein
\ac{XML}-Dokument. Ebenfalls soll man das Ausgabeformat über den "`Accept"'-Wert
des \ac{HTTP}-Headers steuern können. Ziel ist, das Zeile 1 und 3 des
Listings \ref{lst:bsp_abfrage_kon} das gleiche Ergebnis ergeben, genau wie die
Zeilen 2 und 4. Zeile 3 und 4 sind Abfragen von einem Unix-System aus dem Programm
\sona{curl}.
\\
Zunächst ist es geplant, nur ein Abrufen von \ac{SDB} mittels der
\ac{HTTP}-GET-Standardmethode bereitzustellen. Dennoch ist bei dem Design der
\ac{REST}-konformen \ac{API} darauf zu achten, dass sich diese leicht erweitern
lässt. Die Antworten des Servers sind in \ac{UTF-8} codiert.
\begin{figure}[h]
  \centering
    \includegraphics[width=0.75\textwidth]{pic/baum_rest.jpg}
      \caption[Navigationsübersicht durch die REST-konforme API]{Die Darstellung
      zeigt wie mittels der REST-konforme API durch das System euSDB navigiert
      werden soll. Dabei spielt es keine Rolle, ob man durch die Suche direkt
      bei einer Chemikalie startet, oder von der API-Wurzel (root) ausgeht.
      (Darstellung nicht UML-konform)}
      \label{fig:baum_rest}
\end{figure}
\\
Als Einstiegspunkt soll es zwei Möglichkeiten geben. Zum einen den
vordefinierten (\ac{API}-Wurzel) und zum anderen soll es auch mit Hilfe jedes
Suchergebnisses möglich sein, einen Einstieg und somit einen erfolgreichen
Zugriff auf die Daten zu erhalten. Mittels Verlinkungen (Hypermedia) der
einzelnen Ressourcen untereinander soll es dann möglich sein, auf alle relevanten Daten, die im
System zu einem \ac{SDB} gespeichert sind, zugreifen zu können. Abgesehen von
der Wurzel hat jedes Objekt ein Vorgänger-Objekt. Dieses ist ebenfalls verlinkt.
Mit diesen Links kann man durch die gesamte Datenstruktur, die zur Verfügung
steht, navigieren. Abbildung \ref{fig:baum_rest} veranschaulicht, wie die
Verlinkungen und das Navigieren funktionieren.
\begin{lstlisting}[caption={Beispielrequest zur
Steuerung des Paging (Seite 2 mit 50 Einträgen)}, label=lst:bsp_paging,
captionpos=b] 
https://www.euSDB.de/chemicals/?page=2&per_page=50
\end{lstlisting}
\\
Beim Anzeigen von Listen werden bei mehr als 20 Objekten Seiten
ausgegeben (Paging). Dann erhält man, sofern sie verhanden sind, einen
Link auf die vorangegangen und die nachfolgenden Seiten. Mittels
Parameterübergabe lässt sich steuern, wie viele Objekte pro Seite verlinkt werden sollen, und auch das
direkte Springen auf eine Seite ist möglich. Listing \ref{lst:bsp_paging}
veranschaulicht dies anhand eines Beispiels.
\\
Der generelle Aufbau der \ac{URI} ist hierarchisch (Abbildung
\ref{fig:baum_rest}). Es dient dazu, die Treffermöglichkeiten immer weiter
einzuschränken, bis man bei der gewünschten Ressource angekommen ist. Von dort
aus sind dann alle dazugehörigen Anweisungen, Hinweise und Symbole schnell und
direkt erreichbar.

