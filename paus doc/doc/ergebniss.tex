\chapter{Ergebnis}
\label{sec:ErreichteErgebnisse}

\section{Vergleich der Datenbanksysteme zum Speichern der Datensätze
(Access und MySQL)}
\label{sec:vgl_db}

Der Wechsel des Datenbanksystems bringt viele Vorteile mit sich.
\fina{Mircososft}-\sona{Access} ist ein Datenbanksystem, dass nur unter
dem Betriebssystem \sona{Windows} läuft. Somit sind Nutzer des Datenbanksystems
an die Betriebssysteme von \fina{Microsoft} gebunden. Der Erwerb einer
Nutzungslizenz von \fina{Mircososft}-\sona{Access} kostet zur Zeit etwa 360
Euro. Der Umfang an Daten den \sona{Access} verwalten kann ist ebenfalls
eingeschränkt: Eine Datenbank, genauso wie einzelne Tabellen, können nicht
größer als 2 GB werden. Schnell übersteigen \ac{PDF} die
Größe von mehreren MB. Daher könnte eine Datenbank sehr schnell an ihre
Grenzen kommen und dem Datenaufkommen nicht gewachsen sein.
\\
Mit dem Verwenden des Datenbanksystems \sona{MySQL} erhalten die Betreiber von
\sona{euSDB} eine plattformunabhängige und professionelle Lösung, die ihn nicht
dazu zwingen weitere Produkte eines einzigen Herstellers zu benutzen. Die
Möglichkeit das System günstig zu beschaffen und die weitverbreitete Nutzung
erlauben einen effizienten Betrieb. Die Grenzen der Datenmengen, die verwaltet
werden können, sind wesentlich weiter gesteckt als bei einer
\fina{Mircososft}-\sona{Access} Datenbank. Datenbanken unterliegen bei
\sona{MySQL} nur einer physischen Grenze in Form von Festplattengröße bzw. der
Schranke, die das verwendete Betriebssystem beim Verwalten des Speichers setzt.
Tabellen können bis zu 256 TB groß werden und bieten somit eine vielfach höhere
Speicherkapazität als eine \fina{Mircososft}-\sona{Access} Datenbank.
\\
\begin{minipage}[h]
 {\textwidth}
 \centering %wenn tabelle mittig ausgerichtet sein soll
  \begin{tabular}{|p{0.4\textwidth}|p{0.2\textwidth}|p{0.2\textwidth}|l|l|l|}
   \hline
   \ & \textbf{Access} & \textbf{MySQL}\\ \hline \hline 
     Maximale Datenbankgröße & 2 GB & unbegrenzt
     \\ 
     \hline 
     Maximale Tabellengröße & 2 GB & 256 TB  
     \\ 
     \hline  
     Maximale "`Blob"'-Größe & 1 GB & 4 GB  
     \\ 
     \hline 
  \end{tabular}
 \captionof{table}{Vergleich von zwei Datenbanksystemen}
 \label{tab:vgl_dbs}
\end{minipage}
\\
\\
Die Analyse der alten Datenbank ergab ebenfalls, dass diese von Grund auf
strukturell neu entworfen werden muss. Grundprinzipien wie Normalisierung und
Wiederverwendung von Daten wurden nicht genutzt. Außerdem gab es
keine Verknüpfung von Ressourcen, sondern die Gefahren- und Sicherheitssätze
waren einfach in einem Textfeld in der Tabelle abgelegt worden.
\\
Die neue Datenstruktur erlaubt ein schnelles Verstehen der Zusammenhänge, auch
für nicht fachkundige Betreuer der Software \sona{euSDB}. Durch die Einhaltung
strenger Normalisierung werden Redundanzen an Daten vermieden.
Dies ermöglicht auch bei großen Datenmengen speicherplatzschonend 
schnellen Zugriff auf die Daten.
\\
Die Datenstruktur kann problemlos erweitert und verändert werden. Somit lassen
sich zum Beispiel Gesetzesänderungen und Auflagen für \ac{SDB} schnell und ohne
großen Aufwand auf die Datenstruktur übertragen.
\\ 
Der Betreiber ist nicht mehr an ein Betriebssystem gebunden und kann sich frei
entscheiden, welche Plattform er als Serverumgebung nutzen möchte. Dies sind nur
einige Vorteile, die das Datenbanksystem \sona{MySQL} mit sich bringt.

\section{Das PHP-Slim-Framework im Vergleich mit dem Python-Django-Framework}
\label{sec:vgl_frameworks}

Das Vorgängersystem des neu zu entwickelnden Systems ist mit dem
\sona{Slim}\footnote{Die offizielle Webseite des Slim-Framework ist unter
folgender URL zu finden:
\url{http://www.slimframework.com/} zu finden}-Framwork geschrieben.
Dieses Framework ist in der Skript-Sprache \ac{PHP}\footnote{Die
offizielle Webseite von PHP ist unter folgender URL zu finden:
\url{http://php.net/}.
Früher stand die Abkürzung PHP für Personal Homepage Tools} geschrieben.
\\
Bei dem \sona{Slim}-Framework handelt es sich um ein sogenanntes
"`leichtgewichtiges"' Framework. Leichtgewichtige Frameworks dienen dazu, den
Entwicklern einen schnellen Einstieg in eine Materie zu erleichtern. Schnell
sind die funktionalen Grenzen eines solchen Frameworks erreicht.
\\
Dies ist auch der Grund, warum der Entwickler und Betreiber von \sona{euSDB}
nicht mehr weiter mit dem \sona{Slim}-Framework arbeiten möchte. Die
Anforderungen die an die Software \sona{euSDB} gestellt werden, können nicht
mehr ausschließlich durch das \sona{Slim}-Framework bewerkstelligt werden. Das
Erweitern des \sona{Slim}-Frameworks durch den Entwickler ist sehr zeitintensiv
und fordert ein hohes Maß an Fachkunde. Da die Entwicklung und der Betrieb
bisher und auch in Zukunft durch einen Nicht-Informatiker durchgeführt werden
soll, ist es notwendig ein leicht zu erlernendes, allumfassendes und dadurch
mächtiges Framework zu verwenden.
\\
Die Wahl fiel durch den Betreiber auf das \sona{Django}-Framework. Dieses
bringt einen sehr großen Funktionsumfang mit sich. Dieser Funktionsumfang lässt
sich durch weitere Module und Bibliotheken leicht erweitern. Somit stellt es
eine Lösung dar, die leicht zu erlernen ist, und dennoch den Betreiber nicht in
der Funktionalität einschränkt. Die Entwickler des \sona{Django}-Framework
stellen ebenfalls eine ausführliche Dokumentation zur Verfügung. Diese wird mit
jedem Update des Frameworks ebenfalls aktualisiert.
\\
Mit Hilfe der Dokumentation des Frameworks und der Dokumentation der
entwickelten Software kann der Betreiber sich einen Überblick verschaffen, die
Nutzung des Frameworks erlernen und die Plattform seinen eigenen Bedürfnissen
anpassen.

\section{Suchen mit elasticsearch}
\label{sec:elasicsearch_erg}

\sona{elasticsearch} ist eine praktische Unterstützung für das effektive Suchen.
Der Einsatz ist leicht zu erlernen und das System kann schnell und ohne
großen Zeit- oder Arbeitsaufwand an ein \sona{Django}-Projekt angebunden werden.
Die Plattform \sona{euSDB} ist ein relativ kleines Web-Projekt, wenn man es z.B. mit
\href{http://www.youtube.com/}{Youtube} vergleicht. \sona{elasticsearch} ist eine professionelle Lösung, die erheblich
mehr Datenaufkommen durchsuchen kann, als es bei \sona{euSDB} momentan der Fall
ist.
\\
Die Möglichkeiten des Einsatzes auf einer verteilten Serverarchitektur, als
Cluster wird nicht benötigt und nicht genutzt. Auch die damit einhergehende
redundante Datensicherung wird nicht genutzt.
\\
Der Einsatz von \sona{elasticsearch} hat keine negativen Auswirkungen auf das
Gesamtprojekt. Es lässt sich festhalten, dass für den momentanen
Funktionsumfang und den Aufbau des Projekts \sona{elasticsearch}
überdimensioniert ist. Mit \fina{Apache}-\sona{Lucene} oder dem darauf
aufbauenden \fina{Apache}-\sona{Solr} hätten Lösungen für das Einbinden
einer Suchfunktion zur Verfügung gestanden, die wesentlich "`leichter"', also
mit weniger Funktionsumfang, zur Verfügung gestanden hätten.
\\
Der Einsatz von \sona{elasticsearch} hat einen großen Vorteil. Wenn das
System in absehbarer Zeit wachsen würde, oder zu einem auf mehreren
Servern verteilten System umgebaut würde, müsste die Datensuche nicht
komplett neu aufgebaut und überarbeitet werden, sondern kann dynamisch wachsen.
Es müssten lediglich wenige Konfigurationsparameter geändert werden. Das
Verteilen von Daten und das Herstellen einer hohen Datensicherheit würde das
Programm selbstständig übernehmen.

\section{Vergleich der Benutzeroberflächen und Funktionalitäten von euSDB im
alten und neuentwickelten System}
\label{sec:vgl_bof}

Das alte System \sona{euSDB} umfasst bisher nur eine rudimentär
implementierte \ac{REST}-konforme \ac{API}. Diese wurde vom Betreiber zu Testzwecken
entwickelt und wird nur in wenigen Punkten den Anforderungen einer
\ac{REST}-konformen \ac{API} gerecht. Die Themen Hypermedia und eindeutige ID
sind nicht umgesetzt worden.
\\
Die Vergabe von \ac{URI} ist nicht \ac{REST}-konform. Will man alle Chemikalien
eines Herstellers aufgelistet bekommen, ist die zugehörige \ac{URI} nicht
\ac{REST}-konform. Listing \ref{lst:vgl_uri} zeigt die \ac{URI} des alten
Systems in Zeile 1. Die korrekte \ac{URI} ist in Zeile 3 zu sehen.
\begin{lstlisting}[caption={Aufbau der URI im alten System (Zeile 1) und im neuen System (Zeile 3)
}, label=lst:vgl_uri,captionpos=b]
https://localhost:8000/basf/chemicals

https://localhost:8000/producer/basf/chemicals
\end{lstlisting}
\\
Hypermedia, also die Verlinkung von Ressourcen untereinander, ist im
alten System nicht beachtet worden. Folglich wurden die Daten alle in einem
Paket dem Kunden angeboten. Das neue System erlaubt dem Kunden die Daten
auszuwählen, die er benötigt. Somit werden zwar mehrere Abfragen
an das System gestellt, jedoch ist der Kunde flexibler in der Auswahl der Daten,
auf die er zugreifen möchte. Kennt der Kunde die Wurzel der \ac{API}, ist er in
der Lage sich selbständig durch Probieren der angegebenen Links durch das
System zu bewegen. Kunden, die sich im Umgang mit Chemikalien und \ac{SDB}
auskennen, werden ohne genaue Kenntnis der \ac{API}-Spezifikation, schnell die
für sie relevanten Daten finden.
\\
Das alte System war einzig zur Ausgabe von \ac{SDB}
spezialisiert. Eine Auflistung von anderen Ressourcen war nicht vorgesehen. Das
neu entwickelte System bringt in dieser Hinsicht mehr Flexibilität. Die neue
\ac{API} wird den Ansprüchen nach \ac{REST}-Konformität gerecht. Sie erlaubt es,
sich nach den Richtlinien des Hypermedia durch die Ressourcen zu bewegen, ohne
das System genau zu kennen.
\\
Der Kunde kann nun beim Abruf von Daten aus der Datenbank zwischen mindestens
drei Formaten unterscheiden (\ac{XML}, \ac{JSON} und \ac{HTML}). Dies gilt für
den Zugriff auf sämtliche Ressourcen. Im alten System konnte der Zugriff nur
auf den gesamten Datensatz erfolgen. Dabei standen nicht immer die gleichen
Variablen zum Abrufen der \ac{SDB} zur Verfügung. In diesem Bereich ist das
System strukturierter und einheitlicher geworden.
\\
Das neue System \sona{euSDB} umfasst nicht alle Funktionen, die die
\ac{REST}-Konformität bedienen würde. Dies hat seine Gründe darin, dass es nicht
jedem erlaubt sein soll, beliebig die Datensätze zu manipulieren oder neue
hinzuzufügen. Teilweise macht es auch keinen Sinn, gewisse Standardfunktionen zu
implementieren. Dennoch wurde bei der Neuentwicklung stets darauf geachtet, dass
man fehlende Komponenten schnell und leicht in das bestehende System
integrieren kann. Dabei kommt dem Entwickler vor allem der modulare Aufbau
entgegen, der es erlaubt, neue Module in das bestehende System zu integrieren.
\\
Das alte System wurde direkt auf dem Server mittels einer dort installierten
Software verwaltet. Im neuen System steht dem Betreiber und Administrator eine
Weboberfläche zur Verfügung, die er von überall aus dem Web aufrufen kann. Der
Administrator ist also nicht mehr gezwungen, direkt am Server zu arbeiten,
sondern kann auch von entfernten Arbeitsplätzen, oder aus dem Urlaub die
Verwaltung der Plattform durchführen.
\\
Die Ziele des Bachelorprojektes sind im Wesentlichen erreicht. Es sind die
Möglichkeiten des \sona{Django}-Frameworks anhand der prototypischen Umsetzung aufgezeigt
worden. Es eignet sich, um das Projekt nicht nur prototypisch zu
realisieren, sondern auch ein reales Produktivsystem für den globalen Einsatz zu
entwickeln.

\section{Nicht erreichte Ziele}
\label{sec:Nicht_erreichte_ziele}

Einige Ziele und Aufgabenstellungen sind auf Grund von Zeitmangel bisher
außer Acht gelassen worden. Zum einen handelt es sich dabei um einen Dienst, der
den Kunden informiert, sobald sich ein \ac{SDB} ändert, bzw. wenn es
aktualisiert wird. Diese Komponente lässt sich jedoch später entwickeln und in
das existierende System einbinden.
\\
Des Weiteren wurden keine Machbarkeitsstudien zur Klärung der Bezahlung der
Dienstleistungen erstellt. Dies ist jedoch ein zweitrangiges Problem. Es
existiert bereits ein System, dass es ermöglicht, den Kunden
verschiedene Leistungen zur Verfügung zu stellen. Dieses Bezahlsystem kann
erstmal weitergenutzt werden.
\\
Langfristig soll es auch für die Hersteller von Chemikalien finanziell lukrativ
werden, \ac{SDB} auf \sona{euSDB} zu veröffentlichen. Auf welche
Weise man dies sicherstellt, muss ebenfalls geklärt werden und geht in dieselbe
Richtung wie die Machbarkeitsstudien zur Abrechnung. Dabei können prozentuale
Gewinnbeteiligungen ebenso in Betracht gezogen werden, wie
Pauschalenpreise pro Upload gezahlt werden.
