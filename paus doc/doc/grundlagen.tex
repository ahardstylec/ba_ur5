\chapter{Grundlagen}

Einfachheit von Programmstruktur, Datenverarbeitung, Wartung und Nutzung werden
durch lose Kopplung, den Einsatz des \sona{Django}-Frameworks, die
Einhaltung der \ac{REST}-Prinzipien und die Nutzung von \sona{elasticsearch}
angestrebt.

\section{Lose Kopplung}
\label{lose_kopplung_gru}

Teilkomponenten einer Software agieren häufig miteinander. Lose Kopplung
beschreibt den Zustand, dass die einzelnen Komponenten (Teile) der Software von
einander möglichst unabhängig sind: "`So wenige Schnittstellen wie möglich, so
viele wie nötig."' Ändert sich eine Komponente, jedoch nicht deren
Schnittstelle, müssen die gekoppelten Komponenten nicht geändert werden.
\\
Lose Kopplung erleichtert Portierbarkeit, Skalierbarkeit, Erweiterbarkeit und
Wartbarkeit. Als Nachteil stellt sich in einigen Fällen eine schlechtere
Performanz ein. Diese ist bei vielen Systemen, dank der hohen Systemleistung,
aber vernachlässigbar und wird daher billigend in Kauf genommen.
\\
Bei loser Kopplung sind globale Variablen oder öffentliche Attribute zu
vermeiden, damit keine Softwarestati entstehen. Es erlaubt jeder Komponente,
sofern sie kommunikationsbereit ist, ohne Vorbedingungen zu kommunizieren.

\section{Die Programmiersprache Python}
\label{python_gru}

\sona{Python} ist unter anderem eine prozedurale Programmiersprache. Allerdings werden
auch andere Programmierparadigmen von ihr vereint. So ist es zum Beispiel auch
möglich, objektorientiert \cite{BELAGR-2009}\footnote{Diese Quelle findet sich
auch unter folgendem Link als Webseite:
\url{http://openbook.galileocomputing.de/oop/oop_kapitel_01_001.htm#mj311565c5c46a3b13daf9951e408cac80}}
oder funktional \cite{RAIGRI-2009}zu programmieren. 
Der vom Entwickler erzeugte Quellcode wird kompiliert. Der
daraus entstandene Byte-Code kann dann vom Interpreter einer virtuellen
\sona{Python}-Maschine ausgeführt werden. Der Einsatz einer virtuellen Maschine
macht Python plattformunabhängig.
\\
Im Lieferumfang von Python ist eine sehr mächtige Standardbibliothek enthalten.
Sie ermöglicht es dem Entwickler schnell und einfach strukturierte Anwendungen
zu programmieren, die sehr komplexe Aufgaben verrichten und lösen können. Die
Programmiersprache wird gerne für die Prototypentwicklung
eingesetzt.
Auf Grund der Einfachheit, mit der man Software erstellen kann, lassen sich die Prototypen schnell
entwickeln. Auf diese Weise kann man schon früh Designfehler der
späteren Software entdecken und beheben.
\\
\sona{Python} wird unter einer \ac{PSF}-Lizenz vertrieben. Diese Lizenz ist weniger
einschränkend als die \ac{GPL}-Lizenz und
erlaubt es, die \sona{Python}-Software lizenzkostenfrei in große, kommerzielle
Projekte einzubinden.

\\
Ein herausragendes Merkmal von \sona{Python} ist seine hohe Flexibilität. Daher
wird es nicht nur als Programmiersprache in kleinen und
großen Projekten, sondern auch serverseitig im Internet und als
Skriptsprache in anderen Programmiersprachen eingesetzt. Die Wichtigkeit von
\sona{Python} im wachsenden Markt von Embedded-Systemen nimmt immer mehr zu.
\begin{lstlisting}[caption={Beispielfunktion in Python zur Berechnung der Summe
aller Zahlen von 1 bis 5}, label=lst:bsppytsum ,captionpos=b] 
def sum_1_5{self}:
    sum = 0
    for i in range(1,5):
        sum=sum+i
        print sum
    print sum
 \end{lstlisting}
\\
\sona{Python} ist mit dem Anspruch entwickelt worden, für den Entwickler möglichst
einfach zu sein. Dies spiegelt sich schon in seiner Syntax wieder. Diese ist
leicht und schnell zu erlernen. Dabei wird vor allem auf sauber formatierten
Quellcode geachtet. Bei vielen Programmiersprachen, z.B. \sona{Java} oder
\sona{C/C++}, wird ein Befehl durch eine "` ; "' abgeschlossen, der Inhalt einer
Schleife mit "`\{\ldots\}"' eingefasst, in \sona{Python} wird dies über
Einrückung gesteuert (siehe Listing \ref{lst:bsppytsum}).
 \\
Der "`range()"'-Befehl liefert die Zahlen 1 bis 5 in Form einer Liste. Die
"`for"'-Schleife durchläuft mit jedem Iterationsschritt diese Liste und legt die
jeweilige Zahl in die Variable i. Man sieht sofort, dass die einzelnen Befehle
der Funktion nicht durch ein Semikolon abgeschlossen werden und es keine
einfassenden geschweiften Klammern bei der "`for"'-Schleife gibt. Der
"`print"'-Aufruf in der Schleife gibt bei jedem Durchlauf den aktuellen Inhalt
der Variable "`sum"' auf der Konsole aus. Die letzte Zeile Code gibt einmalig
das Endergebnis nach dem vollständigen durchlaufen der Schleife aus. Der
Compiler erwartet zum Einrücken eines hierarchisch untergeordnetem Quellcodes
immer die gleiche Folge von Leerzeichen bzw. Tabulatoren. Ein unter
Programmierern üblicher Standard ist das Einrücken mit vier Leerzeichen.
\\
Mit Hilfe der sogenannten \sona{Python}-\ac{API} ist es möglich, die
Standardbibliothek von \sona{Python} flexibel zu erweitern. Auf diese Weise
können auch Komponenten in anderen Programmiersprachen an \sona{Python}
angebunden werden, z.B. C/C++ Module, die im Gegensatz zu \sona{Python}, sehr
hardwarenah sind.
\\
Die Popularität von \sona{Python} nimmt immer mehr zu. Unter anderem setzt die
\fina{NASA} diese Programmiersprachen ein. Aber auch Internetnutzer
kommen inzwischen tagtäglich mit Produkten auf Basis von \sona{Python} in
Kontakt, z.B. mit \href{http://www.youtube.com/}{Youtube} oder \href{https://www.google.de}{Google}.
Der Entwickler der Programmiersprache \sona{Python}, Guido van Rossum, arbeitet heute selbst
für \fina{Google}.

\section{Webentwicklung mit Django}
\label{sec:django_gru}

"`\sona{Django} ist ein hochwertiges \sona{Python} Web Framework zur schnellen
Entwicklung und für sauberes, praktisches Design."' Dies ist die freie
Übersetzung des Einleitungsatzes der englischsprachigen Internetseite des
\sona{Django}-Projektes: "`Django is a high-level Python Web framework that
encourages rapid development and clean, pragmatic design"' \cite{DJANGO-2013}.
\begin{figure}[ht]
  \centering
    \includegraphics[width=0.8\textwidth]{pic/MTV_1.jpg}
      \caption[Request/Response-Schema in Django]{Ablauf eines Request vom
      Client/Browser an den Server und die Response des Servers an den Client/Browser.}
      \label{fig:mtv}
\end{figure}
\\
Wie man der Abbildung \ref{fig:mtv} entnehmen kann, ist das
\sona{Django}-Framework nach dem \ac{MTV}-Muster (nn der Abbildung \ref{fig:mtv}
grau hinterlegt) aufgebaut. Die View verarbeitet den Request. Über das
entsprechenden Model werden die Daten aus der Datenbank geholt. Die View
verarbeitet dieses Model und dessen Daten mit Hilfe eines Templates und erzeugt
eine Response, die dann über den Webserver an den Client weitergeleitet wird.
Dieser Aufbau ist dem \ac{MVC}-Muster nachempfunden.
\\
Die \ac{MTV}-Architektur, ein zentrales Designmuster der Informatik, entkoppelt
Rohdaten von deren Darstellung, sodass Datenbank und Darstellung der Daten
unabhängig von einander ausgetauscht werden können. Die \ac{MTV}-Architektur
wird von dem \sona{Django}-Framework hervorragend nachgebildet.
\\
Die Komponenten des \sona{Django}-Frameworks sind lose gekoppelt und nach dem
\ac{DRY}-Prinzip aufgebaut. Es gilt zu verhindern, gleiche Funktionen mehrfach
zu implementieren. Änderungen und Fehlerbehebungen müssen dann an genau einer
Stelle im Quellcode durchgeführt werden. Der Code sollte an einer Stelle stehen,
an der man ihn auch erwartet. Dies erleichtert die Verständlichkeit und
Wartbarkeit des Programms.
\begin{figure}[ht]
  \centering
    \includegraphics[width=0.6\textwidth]{pic/ordnerhirachie_basis.jpg}
      \caption[Ordnerhierarchie ohne APP in Django]{Ordnerhierarchie direkt nach
      dem Erzeugen des Projektes euSDB.}
      \label{fig:ordnerhirachie_basis}
\end{figure}
\\
\begin{table}
 \begin{tabular}{|p{0.15\textwidth}|p{0.17\textwidth}|p{0.59\textwidth}|l|l|l|}
 \hline \textbf{Dateiname} & \textbf{Verzeichnis} & \textbf{Aufgabe}\\
     \hline \hline 
     manage.py & Projekt & Das \sona{Django}-Framework enthält eine Datei mit
     dem Namen "`django-admin.py"'. Mit dieser Datei werden fast alle
     administrativen Aufgaben über die Eingabeaufforderung gesteuert. In ihr ist
     auch ein Webserver zum Test des Projektes enthalten. Die "`manage.py"'
     leitet die Befehle an die "`django-admin.py"' weiter und legt vorher das
     Projekt auf den richtigen Pfad, sodass es korrekt vom Framework verarbeitet
     werden kann und setzt die Variable "`DJANGO\_SETTINGS\_MODULE"' so, dass
     sie auf die "`settings.py"' im Konfigurationsverzeichnis zeigt.
     \\ 
     \hline
     \_\_inti\_\_.py & Konfiguration & Diese Datei sorgt dafür, dass das
     Verzeichnis als "`package"' (Paket) erkannt wird. Nur Dateien in
     einem package können in andere Dateien importiert werden.
     \\ 
     \hline
     settings.py & Konfiguration & Hier werden die projektweiten Einstellungen
     gespeichert. Zu diesen gehören zum Beispiel die Anbindung an eine
     Datenbank (siehe Anhang \ref{settings}, Zeile 15 - 24) oder auch Angaben zu
     statischen Verzeichnissen (siehe Anhang \ref{settings}, Zeile 54 - 60).
     \\ 
     \hline
     urls.py & Konfiguration & In dieser Datei werden die möglichen \ac{URL} bzw \ac{URI}  mit
     Hilfe von regulären Ausdrücken abgelegt und mit den entsprechenden Views
     verknüpft. Dies erlaubt es elegant und leicht, höchst flexibele \ac{URI}
     nach eigenen Wünschen und Bedürfnissen zu erzeugen. Es ist auch möglich,
     dynamische \ac{URI} zu bauen.
     \\ 
     \hline
     wsgi.py & Konfiguration & Die Datei dient später beim Veröffentlichen des
     Projektes dazu, dass der Webserver eine Verbindung zu dem Projekt
     herstellen kann und dieses dann für Clients und Browser über das
     Internet zugreifbar ist.
     \\ 
     \hline
  \end{tabular}
 \caption[Dateibeschreibungen eines Django-Projekts]{Beschreibung der Dateien im
 Projektverzeichnis und dem Konfigurationsverzeichnis und deren Aufgaben.}
 \label{tab:datei_bedeutung_i}
\end{table}
Die Entwicklung einer neuen Webanwendung in \sona{Django} erfolgt in einem
sogenannten Projekt. Beim Erstellen eines Projektes erhält man das Projektverzeichnis
mit dem gleichnamigen Konfigurationsverzeichnis (siehe Abbildung
\ref{fig:ordnerhirachie_basis}). Außerdem werden die für das Framework
unbedingt notwendigen Dateien erzeugt. Tabelle
\ref{tab:datei_bedeutung_i} enthält die Kurzbeschreibungen der einzelnen Dateien im
Projekt- und Konfigurationsvereichnis.
\\
Das Projekt kann um viele Anwendungen, sogenannten \ac{APP}, erweitert werden.
Eine \ac{APP} ist dabei in sich abgeschlossen und wird als eigener Ordner im
Projektverzeichnis abgelegt. Hat man beispielsweise eine Webseite mit Gästebuch
und Benutzerverwaltung, dann stellt sowohl das Gästebuch als auch die
Benutzerverwaltung jeweils eine eigene \ac{APP} dar. Auf diese Weise kann man
das Gästebuch einfach entfernen, ohne die Funktionalität der Benutzerverwaltung
zu beeinflussen.
\begin{figure}[ht]
  \centering
    \includegraphics[width=0.6\textwidth]{pic/ordnerhirachie_mit_app.jpg}
      \caption[Ordnerhierarchie mit APP in Django]{Ordnerhierarchie nach dem
      Erzeugen der ersten APP}
      \label{fig:ordnerhirachie_mit_app}
\end{figure}
\\
Der Aufbau des Projektes mit \ac{APP} unterstützt die lose Kopplung der
Anwendungen. Eine \ac{APP} steht dem Projekt erst zur Verfügung, wenn man diese
im Konfigurationsverzeichnis in der "`settings.py"' des Projektes \sona{euSDB}
unter "`INSTALLED\_APPS"' bekannt gemacht hat (siehe Zeile 91 bis 103, Anhang
\ref{settings}).
\\
Eine von der "`manage.py"' erzeugt \ac{APP} besteht dabei immer aus folgenden
Dateien: \_\_init\_\_.py, models.py, tests.py, views.py. Die Ordnerhierarchie
können Sie auch der Abbildung \ref{fig:ordnerhirachie_mit_app} entnehmen. Die
Rolle einzelner Dateien werden im Kapitel \ref{chap:umsetzung} erläutert.
\\
Es kann sinnvoll sein, die entsprechende Struktur zu erweitern, z.B. um eine
\ac{APP}-interne "`urls.py"'. Diese müssen in der projektweiten
"`urls.py"' importiert werden. Auf diese Weise wird eine noch losere Kopplung
der \ac{APP} gewährleistet. Bei der Einbindung muss man nur den Import der
"`urls.py"' (eine Zeile) einfügen und die \ac{APP} in der "`settings.py"'
bekannt machen, und nicht jede einzelne \ac{URL} (mehrere Zeilen) um die App in
das Gesamtprojekt einzubinden. Auch ist es mit einer \ac{APP}-internen
"`urls.py"' einfacher eine \ac{APP} in ein anderes Projekt zu portieren. Bei
sehr Umfangreichen \ac{APP} werden weitere Komponenten benutzt, wie z.B.
Serialisierer oder Administratoren. Diese werden in \ac{APP}-eigenen Dateien
ausgelagert.

\section{Das REST-Prinzip}
\label{sec:rest_prinzip_gru}

\ac{REST}, Representional State Transfer, ist ein Konzept, mit dem
größtmögliche Einfachheit bei Webanwendungen gewährleistet werden soll.
Die Forderungen des Konzepts sind seid der privaten und kommerziellen Nutzung
des Internets bekannt und ergeben sich aus den Problemen bei der
Entwicklung von großen Anwendungslandschaften und der firmenübergreifenden
Integration von Daten:
\begin{itemize}
  \item Lose Kopplung
  \item Interoperabilität
  \item Wiederverwendung
  \item Performance und Skalierbarkeit
\end{itemize}
Diese Probleme versucht man durch Einführung mehrerer Grundprinzipien bei der
Erstellung von Webanwendungen zu lösen bzw. zu umgehen.
Man fasst sie unter dem Begriff \ac{REST} zusammen:
\begin{itemize}
  \item Ressourcen und deren unterschiedliche Repräsentationen
  \item Ressourcen mit eindeutiger Identifikation
  \item Verknüpfungen und Hypermedia
  \item Standardmethoden
  \item Statuslose Kommunikation
\end{itemize}
In den folgenden Abschnitten  wird \ac{REST} an Hand der Grundprinzipien
erklärt.

\subsection{Ressourcen und Repräsentationen}
\label{res_u_rep_gru}

Ressourcen sind die Menge an Daten, die in einer Webanwendung
serverseitig verarbeitet und verwaltet werden. Liegen die Ressorcen
clientgerecht vor, spricht man von Repräsentationen der Ressourcen. Zum Beispiel
gibt es bei einer Benutzerverwaltung für ein Internetkaufhaus die Ressource "`User"'.
Sie wird definiert durch Parameter, im Beispiel durch "`Name"' und
"`E-Mail"' als \ac{HTML}-Repräsentation der Ressource "`User"' (siehe Listing
\ref{lst:format_i_gru}). 
\begin{lstlisting}[
    caption={HTML-Dokument für die Ausgabe eines Users},
    label=lst:format_i_gru, captionpos=b]
<!DOCTYPE html>
<html>
    <head>
    	<title>Benutzerverwaltung - Max Mustermann</title> 
    </head>
    <body>
        <h1>Max Mustermann</h1>
        <p>E-Mail: max.mustermann@bsp.de</p>
    </body>
</html>
\end{lstlisting}
\\
Je mehr Datenformate bzw. Repräsentationen beim Abruf von Ressourcen
unterstützt werden, desto mehr Entwickler werden ein solches System nutzen, und
umso leichter fällt es ihnen. Der Webbrowser ist wohl der weltweit bekannteste
Client für Webanwendungen. Er ist in der Lage, \ac{HTML}-Daten graphisch
aufzubereiten, mit Hilfe von Skripten ist er in der Lage auch andere
Datenformate darzustellen.
\\
Anhang \ref{format_ii_gru} zeigt die "`User"'-Ressource als
\ac{JSON}-Repräsentation, Anhang \ref{format_iii_gru} als
\ac{XML}-Repräsentation.

\subsection{Eindeutige Identifikation}
\label{identification_gru}

Es hat sich als praktisch erwiesen, Ressourcen eindeutige Bezeichner zu geben.
Im Web hat sich für die Vergabe solcher Bezeichner ein einheitliches Konzept
entwickelt. Die einheitlichen Bezeichner werden als \ac{URI} bezeichnet. Mit
Hilfe einer \ac{URI} kann eine Ressource weltweit eindeutig identifiziert werden.
\\
Der Vorteil von einem weltweit anerkannten und genutzten Schema zur Vergabe von
\ac{URI} liegt darin, dass man kein eigenes erfinden muss, stattdessen ein
bereits sehr gut erprobtes verwenden kann, das sehr gut skaliert und
für fast jeden verständlich ist.
\begin{lstlisting}[caption={Beispiel URI eines
Bestellsystems \cite{STETIL-2009a} }, label=lst:bsp_uris_gru ,captionpos=b]
http://example.com/customers/1234 
http://example.com/orders/2007/10/776654
http://example.com/products
\end{lstlisting}
\ac{URI} enthalten alle wesentlichen Elemente zur eindeutigen Identifizierung
einer Ressource oder einer Menge von Ressourcen (Beispiele siehe Listing
\ref{lst:bsp_uris_gru}).
\\
\ac{URL} beschreibt einen Speicherort auf einem Server. \ac{URI}
identifizieren Ressourcen weltweit eindeutig. Der Oberbegriff \ac{URI}
beschreibt auch jede \ac{URL}. Im allgemeinen Sprachgebrauch wird zunehmend
\ac{URI} statt \ac{URL} verwendet.

\subsection{Hypermedia}
\label{hypermedia_gru}

Hypermedia ist die Verallgemeinerung des aus \ac{HTML} bekannten Hypertext. Das
effiziente und sinnvolle Verlinken von Ressourcen und deren Funktionen wird
im Listing \ref{lst:hypermedia_gru} beispielhaft gezeigt und nachfolgend
erklärt.
Die vorgestellte Repräsentation zeigt einen reduzierten Datensatz eines Herstellers.
Zeile 2 gibt den Namen eines Herstellers zurück. Zeile 3 enthält einen Link
auf sich selbst. Dieser kann sehr nützlich sein, wenn die Repräsentation
maschinell verarbeitet wird, oder der Link an andere Personen oder Funktionen
weitergegeben werden muss. Zeile 4 führt dann zu einer Liste mit allen
geführten Chemikalien des Herstellers.
\begin{lstlisting}[caption={Beispiel für ein JSON Dokument eines Herstellers},
label=lst:hypermedia_gru, captionpos=b] { 
    "name": "BASF", 
    "url": "http://localhost:8000/producers/01/.json",
    "chemicals": "http://localhost:8000/producers/basf/chemicals/.json", 
}
\end{lstlisting}
Diese Art der Verlinkung ermöglicht es, fast intuitiv durch die Webanwendung zu
navigieren. Selbst ein Programm, welches diese Daten auswertet, ist sehr leicht
zu programmieren. Der Entwickler eines solchen Programms muss keine Kenntnis
über das serverseitige System haben, da alle für ihn wichtigen \ac{URI} durch
die Repräsentation selbst zur Verfügung gestellt werden. Er kann alle Daten,
den Zusammenhang in dem diese zueinander stehen, den Repräsentationen
entnehmen.
\\
Dem Nutzer können auch die Links auf zur Verfügung stehende Funktionen übergeben
werden, z.B. Registrieren eines neuen Nutzers. Dadurch werden Zustände ohne den
Einsatz von Zustandsvariablen beschrieben. Für den Client können sich jedoch
Zustände und Folgezustände völlig unabhängig vom Server ergeben.
\\
Die Repräsentationen sollen so miteinander verknüpft werden, dass ein
intuitives und leichtes Navigieren zwischen den Ressourcen möglich ist.
Entwickler von Clients können schneller und effizienter programmieren, da sie
nicht zwingend das serverseitige System kennen müssen.

\subsection{Standardmethoden}
\label{standardmethoden_gru}

Das Standardprotokoll im Internet ist \ac{HTTP}. In der
\ac{HTTP}-Spezifikation sind unter anderem sechs Standardmethoden bekannt: GET,
POST, PUT, DELETE, HEAD und OPTIONS. Die Methoden GET, PUT und DELETE sind
laut Spezifikation immer idempotent. Das bedeutet, sollte man mit einer
\ac{URI} wiederholt eine idempotente Standardmethode aufrufen, ist der Zustand
des Systems unverändert. Wird beispielsweise zweimal
ein GET auf eine gleiche \ac{URI} aufgerufen, bekommt man jeweils das gleiche Ergebnis zurück. 
Ruft man mehrfach die DELETE-Funktion auf ein bestimmtes Objekt auf, darf der
Befehl nur einmal ausgeführt werden. Der Aufruf der DELETE-Funktion auf ein
bereits gelöschtes Objekt darf nicht zum Löschen anderer Objekte führen.
\\
POST ist laut der Spezifikation eine nicht idempotente Methode. Das bedeutet es
gibt keine Garantien wie das System bei einem mehrmaligen POST auf
eine URI reagiert. Die Reaktion des Systems hängt von der Implementierung des
Systementwicklers ab.
\\
Die vier Standardmethoden GET, POST, PUT, DELETE sind auch für ein weiteres
Grundprinzip der Webentwicklung von großer Bedeutung. Beim Verwalten
und Bearbeiten von Ressourcen wird mittels des \ac{CRUD}-Prinzips eine
transparente Implementierung vorgeschrieben. Der Nutzer kann sich im allgemeinen
sicher sein, dass die idempotenten Methoden auch so implementiert wurden und ihm
die vier bzw. sechs Standardmethoden bereitgestellt werden, wenn er mit einer
\ac{REST}-konformen Anwendung arbeitet.
\\
Eine natürliche und intuitive Handhabung einer Webanwendung mit
\ac{REST}-konformer \ac{API} setzt voraus, dass alle auf der Website
bereitgestellten Funktionen auch in der \ac{REST}-konformen \ac{API} zur
Verfügung gestellt werden.

\subsection{Statuslose Kommunikation}
\label{kommunikation_gru}

Bei der statuslosen Kommunikation wird der Status des Servers vom Client
gehalten oder als Ressourcenstatus übergeben. Einen Ressourcenstatus lässt
sich anhand eines Onlineshops erklären: Der Status des Kunden wird im
Warenkorb gespeichert und zu einer eigenen Ressource. Der Ressourcenstatus gibt
dem Kunden die Möglichkeit, Lesezeichen zu setzen oder Links darauf zu
verschicken.
\\
Der Verzicht auf einen expliziten Sitzungsstatus hat vor allem Vorteile, wenn es
darum geht, eine Webanwendung skalierbar zu machen. Durch die lose Kopplung
von Client und Server können aufeinanderfolgende Anfragen eines Clients von
verschiedenen Serverinstanzen verarbeitet werden. Während ein Client
arbeitet, ist der Status des Servers für ihn nicht relevant, z.B.
kann der Server gewartet oder repariert werden. Mittels eines Links
werden erst nach erfolgreichem Ausführen der clientseitigen Aufgabe, die
Ergebnisse dem Server mitgeteilt. Hat der Server einen Client mit Informationen
versorgt und die Kommunikation ist abgeschlossen, geht der Server in seinen
Grundzustand zurück und kann andere Anfragen von anderen Clients bearbeiten.
\\
Eine statuslose Kommunikation impliziert ebenfalls eine lose Kopplung. Es
gibt eine klar definierte Schnittstelle, über die die Daten ausgestauscht
werden. Der Server ist völlig unabhängig vom Client und muss keine
clientspezifischen Informationen kennen. Er kann jederzeit mit allen
verarbeitbaren Daten umgehen, ohne dafür in einem zusätzlichen Status befindlich
zu sein.
\\
Dieses \ac{REST}-Prinzip wird genutzt, um eine einfache
Kommunikation zwischen Server und Client sicherzustellen. Die Trennung der
Verantwortlichkeiten für Daten und Operationen sind ein weiteres Ziel. Für den
Datenaustausch sollten Standardformate, wie \ac{XML}, \ac{JSON} usw.,
verwendet werden. Die Kommunikation sollte ebenfalls auf Standardprotokollen
basieren, z.B. \ac{HTTP}. Auf Grund dieser Standards ist der
Informationsaustausch zwischen Client und Server leicht zu realisieren.

\section{Volltextsuche mit elasticsearch}
\label{sec:Volltextsuche_gru}

\sona{elasticsearch}\cite{ELASBV-2013} ist ein auf
\sona{Apache Lucene}\cite{THAPSO-2013} basierender
Suchserver zur Volltextsuche.
Er wurde mit dem Anspruch entwickelt, hoch skalierbar
und dennoch leicht bedienbar zu sein. Dabei sollte ein
Funktionsumfang, wie man ihn von \sona{Apache Lucene} kennt, zur Verfügung
gestellt werden, ohne dabei die Komplexität und die schwierige Bedienung von
\sona{Apache Lucene} auf \sona{elasticsearch} zu übertragen. Die
Entwickler achteten außerdem darauf, dass das System in Form von Modulen
aufgebaut wird, sodass es leicht um neue Module erweitert werden kann.
Ein im Softwarepaket enthaltenes \ac{API}-Modul erlaubt es, den Suchserver zu
konfigurieren und zu bedienen.
\\
\sona{elasticsearch} speichert Daten schemalos und dokumentenorientiert.
Dabei werden die zu indizierenden Daten als {JSON}-Dokument an
\sona{elasticsearch} übergeben. 
\begin{figure}[h]
  \centering
    \includegraphics[width=0.7\textwidth]{pic/aufbau_es_rep.jpg}
      \caption[Aufbau von Nodes in elasticsearch]{Aufbau der Nodes mit Shards
      und Replicas (grau) in einem System mit zwei Nodes. Die Replicas sind die Kopien der Shards eines anderen
      Nodes und werden im Falle eines Node-Ausfalls zu Shards.}
      \label{fig:aufbau_es_node_shard_rep}
\end{figure}
% \begin{figure}[ht]
%   \centering
%     \includegraphics[width=0.8\textwidth]{pic/aufbau_es.jpg}
%       \caption{Aufbau von Nodes und Shardverteilung bei elasticsearch
%       mit einem Node (oben) und zwei Nodes (unten) }
%       \label{fig:aufbau_es_node_shard}
% \end{figure}
\\
Wie \sona{elasticsearch} intern und pysikalisch aufgebaut ist, zeigt Abbildung
\ref{fig:aufbau_es_node_shard_rep}. Jeder Server auf dem \sona{elasticsearch}
läuft, stellt einen sogenannten Node dar. Mehrere Nodes bilden ein Cluster.
Jeder Node besteht aus \sona{Apache Lucene}-Instanzen, den Shards. Die Shards
dienen der Lastverteilung und der parallelen Verarbeitung, wenn mehrere Anfragen
vom Server verarbeitet werden. Bei mehreren Nodes findet eine möglichst
gleichmäßige Verteilung der Shards auf diese statt, jedoch werden einzelne
Shards nicht auf mehrere Nodes aufgeteilt.
Ein logischer Index, die zur Suche bereitgestellten Daten, kann dabei auf
mehreren Shards und mehreren Nodes verteilt abgelegt werden. 
\\
Um eine höhere Ausfallsicherheit zu gewährleisten, enthält ein Node in einem
Verbund von Nodes Replicas. Replicas sind die Kopien der Shards die ein Node
nicht enthält. Graphik \ref{fig:aufbau_es_node_shard_rep} zeigt dies ebenfalls.
Fällt nun ein Node aus, werden die Replicas wieder zu Shards und das System
steht mit allen Daten und Dokumenten zur Verfügung.
\\
Der Aufbau des Systems durch Nodes, Shards und Replicas dient dazu, höchste
Performanz, Effizienz und Ausfallsicherheit zu gewährleisten. Aufgrund der
angesprochenen modularen Aufbauweise und der Unterstützung von Clustern, kann
\sona{elasticsearch} sowohl in verteilten Systemen als auch in nicht verteilten
Systemen effektiv genutzt werden. Entwickler können sehr leicht weitere Module
realisieren, die den Funktionsumfang von \sona{elasticsearch} erweitern oder
verbessern.
\begin{figure}[ht]
  \centering
    \includegraphics[width=0.9\textwidth]{pic/aufbau_es_dat.jpg}
      \caption[Verarbeitung von Daten durch elasticsearch]{Der Index adresses
      verteilt sich auf Node 1 und Node 2. Eines der Dokumente (grau) des Indexes adresses liegt auf Node 1 in
      Shard 3 (Kaiser Platz) und auf Node 2 in Shard 5 (Königsplatz). Der Index
      names liegt auf Node 2 in den Shards 4 (Brigitte Schön) und 5 (Hans
      Schulz)}
      \label{fig:aufbau_es_node_shard_rep_dat}
\end{figure}
\\
Die logische Einteilung ist nicht an die physikalische gebunden. Ein Index kann
zum Speichern von Dokumenten mehrere Shards, sogar auf verschiedenen Nodes
nutzen, andererseits können auf einem Shard auch Dokumente mehrerer Indizes
gespeichert werden (siehe Abbildung \ref{fig:aufbau_es_node_shard_rep_dat}).
\\
Die Logik hinter der Volltextsuche mit \sona{elasticsearch} ist dabei sehr
einfach. Vom System wird eine Tabelle angelegt (Beispieltabelle
\ref{tab:token_dokument_mapping}), die in einer Spalte
alle Wörter (Token) enthält, die in den indizierten Dokumenten vorkommen.
In einer zweiten Spalte werden die Wörter mit allen Dokumenten, in denen das
Wort vorkommt, verknüpft. Sucht man jetzt ein Wort oder einen Wortteil, wird nur
diese eine Tabelle durchsucht und man bekommt sehr schnell alle enthaltenden
Dokumente. Sucht man nach "`Mr"' erhält man nur Dokument 1. Sucht man nach
"`Müller"' ist das Ergebnis nur Dokument 3. Bei der Verknüpfung
von Tokens werden bei der Suchabfrage diese mit der logischen UND-Operation
verknüpft. Die Bearbeitung erfolgt serverseitig, genaugenommen bitweise. Das hat
zur Folge, dass die Suchabfrage "`Mr. Müller"' keinen Treffer ergibt.
Die Suchanfrage "`Michael Jackson"' liefert als Suchergebnis die Treffer in
Dokument 2 und 3. Dokument 1 wird nicht berücksichtigt.
\begin{minipage}[h]
 {\textwidth}
 \centering %wenn tabelle mittig ausgerichtet sein soll
  \begin{tabular}{|p{0.3\textwidth}|p{0.3\textwidth}|l|l|}
   \hline
   \textbf{Token} & \textbf{Dokument(e)}\\ \hline \hline 
     Mr & 1
     \\ 
     \hline 
     Jackson & 1+2+3
     \\ 
     \hline 
     verstarb & 1
     \\ 
     \hline 
     vor & 1
     \\ 
     \hline 
     wenigen & 1
     \\ 
     \hline 
     Jahren & 1
     \\ 
     \hline 
     Michael & 2+3
     \\ 
     \hline 
     war & 2+3
     \\ 
     \hline 
     ein & 2+3
     \\ 
     \hline 
     berühmter & 2
     \\ 
     \hline 
     Künstler & 2
     \\ 
     \hline 
     Müller & 3
     \\ 
     \hline 
     Fan & 3
     \\ 
     \hline 
     von & 3
     \\ 
     \hline 
  \end{tabular}
 \captionof{table}[Aufbau eines Beispielindex in elasticsearch]{Tabelle, die den
 Wörtern (Tokens) Dokumente zuweist.
 Die Tabelle enthält unter anderem die drei Sätze: "`Mr. Jackson
                   verstarb vor wenigen Jahren."'(Dokument 1) "`Michael Jackson
                   war ein berühmter Künstler."'(Dokument 2) "`Michael Müller
                   war ein Fan von Micheal Jackson."'(Dokument 3)}
 \label{tab:token_dokument_mapping}
\end{minipage}

