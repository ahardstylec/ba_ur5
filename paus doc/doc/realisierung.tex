\chapter{Realisierung}
\label{chap:umsetzung}

\section{Datenbank}
\label{sec:Datenbank_rel}

Das \sona{Django}-Framework nimmt dem Entwickler einige Aufgaben ab. So kann es
unter anderem auch eine Datenbank gemäß den Einstellungen und Klassen erzeugen, die
bei der Umsetzung verwendet und eingesetzt werden. Die Einstellungen, die die
Kommunikation mit der Datenbank ermöglichen, werden in der "`settings.py"'
(siehe Anhang \ref{settings}) im Konfigurationsverzeichnis vorgenommen. Die
Zentrale Einheit zur Verwaltung und Nutzung einer Datenbank ist in jeder \sona{Django}-\ac{APP} die "`models.py"';
in Kapitel \ref{sec:django_gru} wurde bereits die Ordnerhierarchie und der Aufbau
von \sona{Django}-Projekten vorgestellt.
\begin{lstlisting}[caption={Einstellungen eines Django-Projektes zur
Ansteuerung eines MySQL-Datenbanksystems
}, label=lst:anbindung_db_rea,captionpos=b] 
DATABASES = {
    'default': {
        'ENGINE': 'django.db.backends.mysql',
        'NAME': 'euSDB',
        'USER': 'eusdb',
        'PASSWORD': 'XXX',
        'HOST': '',
        'PORT': '',
    }
}
\end{lstlisting}
\\
Die Einträge in der "`settings.py"', die hier auch noch mal in Listing
\ref{lst:anbindung_db_rea} aufgeführt sind, erlauben den Zugriff auf ein lokales
\sona{MySQL}-Datenbanksystem. In Zeile 3 wird der Typ der verwendeten Datenbank
und die Komponente ausgewählt, die die Verbindung zur Datenbank aufbauen wird.
In diesem Fall ist eine Standardklasse des \sona{Django}-Framework angegeben.
In Zeile 4 steht der Name der Datenbank, danach folgen der Benutzername und
dessen Passwort in Klartext (hier in Form von "`XXX"' dargestellt). "HOST"' und
"`PORT"' werden je nach Datenbanksystem automatisch eingefügt und können in
diesem Fall frei gelassen werden.
\begin{lstlisting}[caption={Ausschnitt aus der
Datei "`models.py"' der chemicals-APP, die in der Datenbank durch die Tabelle
"`chemicals\_chemicals"' und allen dazugehörigen Matchingtabellen dargestellt
wird }, label=lst:chemical_model,captionpos=b]
class Chemical(models.Model):
    """Chemical model."""
    id = models.AutoField(u'id', primary_key=True, editable=False)
    name = models.CharField(u'Name', max_length=255, default='')
    company = models.ForeignKey(CompanyData,
        verbose_name=u'Hersteller-Firma', related_name='chemicals')
    hazard = models.ManyToManyField(Hazard, verbose_name=u'Hazard Symbol')
    ghs = models.ManyToManyField(Ghs, verbose_name=u'GHS Gefahren Kennung')
    eu = models.ManyToManyField(Eu, verbose_name=u'EU Gefahren Kennung')
    p_phrase = models.ManyToManyField(P_phrase, verbose_name=u'P-Satz')
    s_phrase = models.ManyToManyField(S_phrase, verbose_name=u'S-Satz')
    r_phrase = models.ManyToManyField(R_phrase, verbose_name=u'R-Satz')
    h_phrase = models.ManyToManyField(H_phrase, verbose_name=u'H-Satz')
    slug_chemical = models.SlugField(unique=True)
    previous_data = models.OneToOneField('self', blank=True,
        null=True, related_name='Vorgaenger_von')
    replacement_data = models.OneToOneField('self', blank=True,
        null=True, related_name='Nachfolger_von', editable=False)
    last_update = models.DateTimeField(editable=False)
\end{lstlisting}
\\
In der Datei "`models.py"' einer jeden \ac{APP} werden die einzelnen Tabellen in
Form von Klassen abgelegt. Die Variablen dieser Klassen findet man in der
Datenbank als Felder der entsprechenden Tabellen wieder. Das Listing
\ref{lst:chemical_model} zeigt die Klasse, die die Tabelle
"`chemicals\_chemicals"' mit all ihren Matchingtabellen auf andere Tabellen darstellt. Das erste "`chemicals"' steht dabei
für die \acc{APP}, der die Tabelle zuzuordnen ist, das zweite für die
eigentliche Komponente. Es existieren noch die Tabellen "`chemicals\_p\_phrase"',
"`chemicals\_ghs"', \ldots. Die Matchingtabellen werden nach der \ac{APP}, der
Hauptkomponente und der zu verknüpfenden Komponente benannt. Es entstehen
Tabellennamen wie "`chemicals\_chemicals\_p\_phrase"',
"`chemicals\_chemicals\_ghs"', \ldots.
\\
Man erkennt an Listing \ref{lst:chemical_model} sehr gut, wie die Komponenten
mit Chemikalien verknüpft werden. Der Hersteller wird durch das Aufnehmen
seines Schlüssels als Fremdschlüssel mit der Tabelle "`chemicals\_chemicals"'
verknüpft (siehe Zeilen 5 und 6). Andere Komponenten werden in Form von
Many-To-Many Beziehungen verknüpft (siehe Zeilen 7 bis 13).
\\
Dabei existieren im \sona{Django}-Framework schon Basisklassen, für die
wichtigsten Datentypen und für das Verknüpfen von Komponenten. Diese sind in der
Klasse "`Models"' gespeichert und können verwendet werden. Es ist auch möglich,
eigene Feldtypen zu definieren, die das bestehende Angebot an
Datentypen erweitern.
\\
Hat man die "`Models"' definiert und eine Verbindung zu einem Datenbanksystem
erstellt, kann das \sona{Django}-Framework über die Eingabe der Befehle aus
Listing \ref{lst:erstellung_db_rea} selbstständig eine Datenbank mit den
Tabellen erzeugen. Zeile 1 überprüft zunächst, ob Syntaxfehler vorliegen. Zeile
2 erstellt die \ac{SQL}-Befehle und speichert sie zwischen. Zeile 3 sendet die
\ac{SQL}-Befehle an das Datenbanksystem und sorgt somit für das direkte
Erstellen der Datenbank.
\begin{lstlisting}[caption={Befehlfolge zur Eingabe in ein Linux-Terminal
direkt auf dem Server zur Erstellung einer Datenbank in einem Datenbanksystem
mit allen Tabellen der APP chemicals
}, label=lst:erstellung_db_rea,captionpos=b]
python manage.py validate
python manage.py sqlall chemicals
python manage.py syncdb
\end{lstlisting}
\\
Die Datei mit allen Model-Klassen ist im Anhang \ref{model_py} angefügt.
Betrachtet man diese Datei, stellt man fest, dass diese eine Subklasse enthält.
In dieser wird bestimmt, wie der Name des bzw. der Objekte ist. Ebenfalls gibt
es Funktionen, die die Handhabung der Objekte erleichtern sollen. Mit der in den
Zeilen 19 und 20 von Listing \ref{model_py} definierten Methode
"`\_\_unicode\_\_"' wird eine Standardtextausgabe eines Objektes erzeugt. Sie kann
sowohl im Terminal, als auch auf Webseiten verwendet werden.
\\
Da es manchmal nicht reicht, Objekte einfach in einer Datenbank abzulegen, wird
in der Methode "`save"' (z.B. Listing \ref{model_py}, Zeile 22 bis 29) die
Speichermethode überladen. Es wird ein Erstellungsdatum gesetzt, wenn das Objekt
vom Datenbanksystem noch keine "`ID"' zugewiesen bekommen hat. Ist eine "`ID"' schon
verhanden und wird nur eine Aktualisierung bzw. Veränderung an einem
Datensatz durchgeführt, ist es nicht mehr notwendig das Erstellungsdatum zu
verändern (z.B. Listing \ref{model_py}, Zeile 23 und 24). 

\section{REST-Schnittstelle}
\label{sec:REST-Schnittstelle_rel}

\subsection{Aufbau der URI}
\label{sec:aufbau_uri}

Der Aufbau der \ac{URI} wird hierarchisch durchgeführt. Das bedeutet, dass nicht
alle Funktionen direkt über die Wurzel der \ac{API} aufgerufen werden, sondern
sich aus gewissen Zusammenhängen zwischen den Ressourcen ergeben.
\begin{lstlisting}[caption={Ein Ausschnitt aus der Datei "`urls.py"' der APP "`chemicals"'. Anhand
    dieser lässt sich der Aufbau der hierarchischen URI
    nachvollziehen}, label=lst:urls, captionpos=b] 
urlpatterns = patterns('chemicals.views',
  url(r'^$', 'api_root', name = 'api_root'),
  url(r'^producers/$', views.ProducerList.as_view(), name = 'producer_list'),
  url(r'^producers/(?P<id_comp>[-\w]+)/$', views.ProducerDetail.as_view(), 
      name = 'companydata-detail'),    
  url(r'^producers/(?P<id_comp>[-\w]+)/chemicals/$', 
      views.ChemicalList.as_view(), name = 'chemical_list'),
  url(r'^producers/(?P<id_comp>[-\w]+)/chemicals/(?P<id_chem>[-\w]+)/$', 
      views.ChemicalDetail.as_view(), name = 'chemical-detail'),   
  url(r'^producers/(?P<id_comp>[-\w]+)/chemicals/(?P<id_chem>[-\w]+)
      /hazards/$',
      views.HazardList.as_view(), name = 'hazard_list'),
  url(r'^producers/(?P<id_comp>[-\w]+)/chemicals/(?P<id_chem>[-\w]+)
      /hazards/(?P<id_haz>[-\w]+)/$', views.HazardDetail.as_view(), 
      name = 'hazard-detail'),
  url(r'^producers/(?P<id_comp>[-\w]+)/chemicals/(?P<id_chem>[-\w]+)
      /p_phrases/$', 
      views.P_phraseList.as_view(), name = 'p_phrase_list'),
  url(r'^producers/(?P<id_comp>[-\w]+)/chemicals/(?P<id_chem>[-\w]+)
      /p_phrases/(?P<id_p_ph>[-\w]+)/$', views.P_phraseDetail.as_view(), 
      name = 'p_phrase-detail'),
   )
\end{lstlisting}
\\
Die im Listing \ref{lst:urls} gezeigten \ac{URI}-Muster realisieren diese
Hierarchie. Direkt unter der Wurzel befinden sich die "`producer"'. Hierarchisch
unter einem "`producer"' befinden sich die mit ihm verknüpften "`chemicals"',
unterhalb der "`chemicals"' befinden sich dann die weiteren Ressourcen.
\\
Der Aufbau eines Befehls zur Verarbeitung von Anfragen ist dabei fast immer
identisch. Zeile 8 bis 10 des Listings \ref{lst:urls} ist wie folgt aufgebaut:
"`url(\ldots)"' erzeugt ein neues Muster für eine \ac{URI}; dann folgt ein
regulärer Ausdruck, der den genauen Aufbau der \ac{URI} definiert. Die
Variablen sind daran zu erkennen, dass sie von "`<\ldots>"' eingefasst
sind. Im Beispiel sind folgende Variablen angegeben:
"`<id\_comp>"' und "`<id\_chem>"', es sind die ID der entsprechenden
Ressourcen.
\\
Nach dem folgenden Komma erfolgt der Aufruf der verarbeitenden "`view"'. Views
können objektorientiert oder funktional programmiert werden. Die im Listing
\ref{lst:urls} in Zeile 2 gezeigte \ac{URI} zur \ac{API}-Wurzel ist durch
einen Funktionsaufruf, die anderen durch den Aufrufen der
objektorientierten "`view"'-Klasse realisiert worden.
\\
Der Aufruf der Detailansicht einer einzigen Ressource ist über die Verwendung
der ID eindeutig geregelt. Es gibt keine Mehrfachverwendung der ID und
es ist der Eindeutigkeit im Bezug auf \ac{URI} genüge getan.

\subsection{Serialisierung von Ressourcen}
\label{sec:serialisierung}

\begin{lstlisting}[caption={Serialisierer-Klassen zur Serialisierung von
Chemikalien. Sowohl für die Detailansicht, als auch für die Listenansicht
(Ressource: Producer)
}, label=lst:serialisierer, captionpos=b] 
class CompanyData_List_Serializer(serializers.HyperlinkedModelSerializer):
    url = serializers.HyperlinkedIdentityField(lookup_field='slug_company',
        view_name='companydata-detail',)
    class Meta:
        model = CompanyData
        fields = ('name', 'url')
    def restore_object(self, attrs, instance=None):
        return CompanyData(**attrs)

class PaginatedCompanyDataListSerializer(PaginationSerializer):
    class Meta:
        object_serializer_class = CompanyData_List_Serializer

class CompanyData_Detail_Serializer(serializers.HyperlinkedModelSerializer):
    url = serializers.HyperlinkedIdentityField(lookup_field='slug_company',
        view_name='companydata-detail',)
    chemicals = serializers.SerializerMethodField('get_producer_chemicals')
    def get_producer_chemicals(self, obj):
        return reverse('chemical_list', args=[obj.slug_company, 
            self.context['format']], request=self.context['request'])
    class Meta:
        model = CompanyData
        fields = ('url', 'name', 'chemicals', 'previous_data', 
            'replacement_data')
    def restore_object(self, attrs, instance=None):
        return CompanyData(**attrs)
\end{lstlisting}
\\
Ressourcen werden zur Verarbeitung durch Views serialisiert.
Sie liegen dann nicht mehr als strukturierter Datensatz vor, sondern in einer
sequenziellen Form.
\\
Zur Serialisierung von Ressourcen wird die chemicals-\ac{APP} um
die Datei "`serializer.py"' erweitert. Diese Datei enthält alle
Serialisierer-Klassen, die zur Serialisierung von Daten notwendig sind.
Listing \ref{lst:serialisierer} zeigt die Serialisierer-Klassen zur
Serialisierung der Chemikalien sowohl in Listenform, als auch für die
Detailansicht.
\\
An Hand des Beispiels (Listing \ref{lst:serialisierer}) sieht man, dass nur die
wichtigsten Informationen einer Ressource zur Darstellung für die
Listenübersicht (Zeile 1 bis 8) übergeben werden, Name und \ac{URI} (Zeile 6).
\\
Erst in der Detailansicht (Zeile 14 bis 26) werden mehr Informationen zur
Verfügung gestellt. Die Ausgabe der Informationen kann ebenfalls
angegeben werden (Zeile 23 und 24).
\\

\subsection{Funktionsweise der Views}
\label{sec:views}

\begin{lstlisting}[caption={View-Klassen zur Bearbeitung von Anfragen an das
System und zum Erstellen der entsprechenden Antworten (Ressource: Producer)
}, label=lst:view,captionpos=b] 
class ProducerList(APIView):
    renderer_classes = (TemplateHTMLRenderer, UnicodeJSONRenderer, XMLRenderer)  
    authentication_classes = (SessionAuthentication, BasicAuthentication)
    permission_classes = (IsAuthenticated,)
    def get(self, request, format=None):
        companys = CompanyData.objects.all().filter(replacement_data=None)
        paginate_by = 3
        paginator = Paginator(companys, paginate_by)
        page = request.QUERY_PARAMS.get('page')
        try:
            obj = paginator.page(page)
        except PageNotAnInteger:
            obj = paginator.page(1)
        except EmptyPage:
            obj = paginator.page(paginator.num_pages)
        serializer = PaginatedCompanyDataListSerializer(obj, 
            context ={'request': request, 'format': format})
        if (format == 'html') | (format == None):
            return Response({'object_list': obj}, 
                template_name='chemicals/producer_list.html')
        return Response(serializer.data)

class ProducerDetail(APIView):
    renderer_classes = (TemplateHTMLRenderer, UnicodeJSONRenderer, XMLRenderer)  
    authentication_classes = (SessionAuthentication, BasicAuthentication)
    permission_classes = (IsAuthenticated,)
    def get_object(self, slug_company):
        try :
            return CompanyData.objects.get(slug_company=slug_company)
        except CompanyData.DoesNotExist:
            raise Http404
    def get(self, request, slug_company, format=None):
        companys = self.get_object(slug_company)
        serializer = CompanyData_Detail_Serializer(companys, 
            context ={'request': request, 'format': format})
        if (format == 'html') | (format == None):
            return Response(serializer.data, 
                template_name='chemicals/producer_detail.html')
        return Response(serializer.data)
\end{lstlisting}
\\
Die Views erzeugen die Repräsentationen, die der Kunde von \sona{euSDB} auf
seinem Client oder Webbrowser zur Verfügung gestellt bekommt. Dabei wird
unterschieden, ob es eine Standardausgabe (\ac{JSON} oder \ac{XML}) oder eine
graphisch aufbereitete Ausgabe (\ac{HTML} oder \ac{PDF}) ist.
\\
Im ersten Schritt werden der View Renderer-Klassen zur Verfügung gestellt. Dabei
handelt es sich um Standardklassen und Funktionen, die in der Lage sind, Daten
in das entsprechende Format zu überführen. Darauf folgen die Authentifizierungs-
und Rechtevergabe-Klassen. Das ist notwendig, um zu verhindern, dass Unbefugte
Zugriff auf das System bekommen.
\\
Zum Implementieren der Standardmethoden des \ac{HTTP} sind für Entwickler
mehrere Methodennamen ("`get"', "`post"', "`head"', \ldots) reserviert. In dem
Beispiel (Listing \ref{lst:view}) sind die "`get"'-Methoden (Zeile 5 bis 21 und
32 bis 39) realisiert.
Neben den für die Standardmethoden reservierten Funktionen können auch weitere
Funktionen in die Klassen implementiert werden. Zeile 27 bis 31 des Listings
\ref{lst:view} rufen die Daten aus der Datenbank ab, ist das nicht erfolgreich,
erzeugen sie eine Standard-Fehlermeldung.
\\
Die Standardausgaben müssen nicht graphisch aufbereitet werden. Sie haben den
Zweck, möglichst kompakt die Daten dem Kunden zur maschinellen
Weiterverarbeitung zur Verfügung zu stellen. Dabei ist ein Seitenlayout nicht von Bedeutung und
würde bei einer schnellen Verarbeitung stören. Der Datensatz wird sowohl in
der Listenübersicht (Listing \ref{lst:view} Zeile 1 bis 21), als auch in der
Detailansicht (Listing \ref{lst:view} Zeile 23 bis 39) jeweils serialisiert.
Dabei werden die im vorigen Kapitel \ref{sec:serialisierung} vorgestellten
Serialisierer verwendet (Zeilen 16 und 17 bzw. 34 und 35).
\\
Nachdem die Daten serialisiert wurden, werden sie aufbereitet und dem Kunden als
Antwort zugeschickt. Dabei werden die Daten in den Formaten \ac{JSON} und
\ac{XML} direkt an den Kunden versandt. Sollen die Daten jedoch als
\ac{HTML}-Dokument ausgegeben werden, generiert die View mit Hilfe eines
Templates das Antwortdokument (z.B. Zeile 18 bis 20).

\subsection{Hypermedia zwischen den Ressourcen}
\label{sec:hypermedia}

Die Grundlagen über Hypermedia sind im Kapitel \ref{hypermedia_gru} zu finden.
Hier wird beschrieben, wie die Verlinkungen zwischen den Ressourcen realisiert
wurden.
\\
Die Models wurden, wie im Abschnitt \ref{sec:serialisierung} beschrieben,
serialisiert, sodass sie von den Models verarbeitet und verwendet werden
können. Im Beispiel (Listing \ref{lst:serialisierer}) wird in den Zeilen 2 und 3
die Variable "`url"' definiert. Diese dient dazu, in der Listenansicht zu
jeder Instanz einer Ressource einen Link zu erzeugen.
\begin{lstlisting}[caption={Definition der Variablen "`url"' aus dem Serialisierer einer Liste} 
, label=lst:url_def,captionpos=b] 
    serializers.HyperlinkedIdentityField(lookup_field='slug_company',
        view_name='companydata-detail',)
\end{lstlisting}
\\
Der Befehl aus Listing \ref{lst:url_def} funktioniert wie folgt:
Das "`HyperlinkedIdentityField"' erzeugt einen Link auf eine Ressource. Dabei
wird als Identifizierungsfeld (lookup\_field) ein ID-Feld (id\_\ldots)
der zu verlinkenden Ressource verwendet. Am Schluss wird dann die
aufzurufende View angegeben.
\begin{lstlisting}[caption={Definition der Variablen "`url"' und
"`chemicals"' aus dem Serialisierer der Detailansicht eines Herstellers} 
, label=lst:hypermedia,captionpos=b] 
    url = serializers.HyperlinkedIdentityField(lookup_field='id_comp',
        view_name='companydata-detail',)
    chemicals = serializers.SerializerMethodField('get_producer_chemicals')

    def get_producer_chemicals(self, obj):
        return reverse('chemical_list', args=[obj.id_comp, 
            self.context['format']], request=self.context['request'])
\end{lstlisting}
\\
Bei der Vergabe von Links in der Detailansicht gibt es zwei Links. Einer
verweist auf sich selbst ("`url"') und einer, im Beispiel (Listing
\ref{lst:hypermedia}) auf die Chemikalien ("`chemicals"'). Der Link auf sich
selbet kann clientseitig als Lesezeichen oder zur Weitergabe verwendet werden. Die Variablen werden auch im Serialisierer
initialisiert (Listing \ref{lst:hypermedia}, Zeile 1 bis 3). Es ist notwendig,
den Link auf eine andere Ressource selbst zu erstellen, da das System eigentlich
keine hierarchischen \ac{URI} unterstützt. Dies übernimmt die Funktion aus den
Zeilen 5 bis 7. Hier wird mittels des "`reverse"' Befehls eine Link erstellt.
Diesem werden das vom Kunden angeforderte Datenformat und der Hersteller an
die aufzurufende View ("`chemical\_list"') übergeben. Am Ende entsteht eine absolute \ac{URI}.
\\
Ohne den genauen Aufbau des Systems zu kennen, kann sich der Kunde über die
Verlinkungen und bereits bekannten Links durch das System bewegen.

\subsection{Paging in den Listenansichten}
\label{sec:paging}

Das Paging, also das Einteilen von Listen in mehrere Pakete (Seiten), wird
sowohl von der View, als auch dem Serialisierer gesteuert. Dabei teilt der
Serialisierer den Datensatz auf (Listing \ref{lst:paging}).
\begin{lstlisting}[caption={Einteilen der Seiten durch den Serialisierer (Ressource: Producer) }
, label=lst:paging,captionpos=b]
class PaginatedCompanyDataListSerializer(PaginationSerializer):
    class Meta:
        object_serializer_class = CompanyData_List_Serializer
\end{lstlisting}
\\
Die von dem "`CompanyData\_List\_Serializer"' gelieferten Daten werden durch
den "`PaginationSerializer"', einer Standardmethode, in Seiten eingeteilt.
\\
In der View kann man angeben, wie viele Einträge pro Seite zu führen sind, und
welche Seite gerade zurückgeliefert werden soll.
\begin{lstlisting}[caption={Steuerung und Validierung des Paging in der View (Ressource: Producer) }
, label=lst:paging_view,captionpos=b]
        paginate_by = 25
        paginator = Paginator(companys, paginate_by)
        page = request.QUERY_PARAMS.get('page')
        try:
            obj = paginator.page(page)
        except PageNotAnInteger:
            obj = paginator.page(1)
        except EmptyPage:
            obj = paginator.page(paginator.num_pages)
\end{lstlisting}
\\
In Listing \ref{lst:paging_view} wird in Zeile 1 ein Standardwert für die
Anzahl an Einträgen pro Seite angegeben. Danach werden die Daten von der
Datenbank angefordert und unter "`page"' gespeichert. In den Zeilen 4 bis 9 wird
dann überprüft, ob die Anfrage korrekt ist und beantwortet werden kann; z.B. ob
für die Seitenzahl ein Integer-Wert angegeben oder eine leere Seite
angefordert wurde. Sollte eine ungültige Anfrage gestellt werden, wird die erste
Seite als Antwort bereitgestellt.

\section{Suchfunktion}
\label{sec:Suchfunktion_rel}

\begin{lstlisting}[caption={Die Datei "`search\_index.py"' verwaltet den Index und die zu indizierenden
    Daten
} 
, label=lst:search_index,captionpos=b]
class ChemicalIndex(indexes.SearchIndex, indexes.Indexable):
    text = indexes.EdgeNgramField(document=True, use_template=True)
    name = indexes.CharField(model_attr='name')
    cas = indexes.CharField(model_attr='cas')
    ec_number = indexes.CharField(model_attr='ec_number')
    ec_index_number = indexes.CharField(model_attr='ec_index_number')
    hill_formula = indexes.CharField(model_attr='hill_formula')
    pub_date = indexes.DateTimeField(model_attr='pub_date')

    def get_model(self):
        return Chemical

    def index_queryset(self, using=None):
        return self.get_model().objects.
            filter(pub_date__lte=datetime.datetime.now())
\end{lstlisting}
\\
Die Suchfunktion ist in einer eigenen \ac{APP} ausgelagert. Die
"`search"'-\ac{APP} stellt die Verbindung zu \sona{elasticsearch} her und fügt
die Daten automatisch in die Suchdatenbank ein.
\\
Die "`search"'-\ac{APP} enthält neben den automatisch von \sona{Django}
erstellten Dateien die Datei "`search\_index.py"' (Listing
\ref{lst:search_index}).
\\
In den Zeilen 2 bis 8 werden die zu indizierenden Werte einer Chemikalie
definiert. Es handelt sich in den meisten Fällen um Textfelder (CharField).
Dabei werden Werte verwendet, die es erlauben, Chemikalien möglichst eindeutig
zu identifizieren. Auf diese Weise wird die Suchdatenbank schlank und
schnell gehalten.
\\
Die Klasse "`ChemicalIndex"' hat eine Funktion, die es ihr erlaubt
auf Chemikalien zuzugreifen (Zeile 10 und 11). Diese Funktion ist
notwendig, damit man schnell Zugriff auf zu aktualisierende Datensätze bekommt.
Das Abrufen von Suchergebnissen wird mit Standardmethoden der Bibliothek
"`haystack"' durchgeführt.
\\
Die "`haystack"'-Bibliothek ist eine speziell zur Anbindung von
\sona{elasticsearch} an \sona{Django}-Projekte entwickelte Programmbibliothek.
Sie verfügt über eine Grundfunktionalität, die der Entwickler original nutzen
kann, oder sie gegebenenfalls anpassen muss.
\\
Zu der Suchfunktion wird ein Template erstellt. Dieses enthält ein
Suchfeld, um die Anfragen einzugeben. In einer Textdatei wird 
angegeben, in welcher Reihenfolge und welche Felder
durchsucht werden. Die View und auch der Aufbau des Formulars muss
nicht durch den Entwickler, es sei denn, er möchte ein individuelles Formular
haben, definiert werden. Die "`haystack"'-Bibliothek enthält die notwendigen
Views und Formulare und stellt sie dem Entwickler zur Verfügung.



