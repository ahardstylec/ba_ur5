\chapter{Fazit}

\section{Ausblick}
\label{sec:ausblick}

Die Plattform \sona{euSDB} kann noch um viele Projekte und Funktionen erweitert
werden. Zunächst gilt es, die noch offenen Ziele aus Kapitel
\ref{sec:Nicht_erreichte_ziele} zu erreichen.
\\
Es macht Sinn die \ac{REST}-konforme \ac{API} in einer eigenen \ac{APP} auszulagern.
Bisher ist sie in die Kern-\ac{APP} eingegliedert. Das Auslagern bietet den
Vorteil, dass man leicht eine neue Version der \ac{API} hinzufügen kann, ohne
das die bestehende \ac{API} ungültig wird. Für Nutzer der \ac{API} ist dies sehr
wichtig, denn diese haben nicht immer das Interesse ihren Client neu
programmieren, sobald es eine neue Version der \ac{API} gibt. Damit der Client weiterhin nutzbar
bleibt, ist es notwendig, parallel mehrere Versionen der \ac{API} zu betreiben.
Dies kommt auch dem Prinzip der losen Kopplung entgegen und erlaubt
schnelles Ein- und Ausbinden der \ac{API}.
\\
Das wohl größte Projekt wird die Migration der Daten von dem alten auf das neue
System. Dabei gilt es viele Probleme zu lösen. Leider sind die Daten nicht alle
konsistent und einheitlich abgelegt. Es ist zu vermuten das die ca. 440 000
Datensätze nur teilweise automatisiert migriert und aufbereitet werden können.
Bei vielen Details wird man um eine manuelle Bearbeitung und Prüfung nicht
herumkommen.
\\
Die Regeln zur Notation von S-Sätzen sind folgende: 1. S-Sätze werden in der
Zahlencodierung durch ein Semikolon getrennt. 2. Treten S-Sätze in festen
Kombinationen auf, werden sie durch einen Schrägstrich getrennt.
\begin{lstlisting}[caption={Beispieldaten, die aus formalen Gründen nicht korrekt sind und einer
    Überarbeitung bedürfen, bevor sie in das neue System übernommen werden
    können}
, label=lst:bsp_inkon,captionpos=b]
S-Saetze:
26/36/37/39
\end{lstlisting}
\\
Listing \ref{lst:bsp_inkon} zeigt die inkonsistente Darstellung der S-Sätze
einer Chemikalie in der alten Datenbank. Die S-Sätze 26 und 39 sind einzeln,
36/37 treten als feste Kombination auf. Die richtige Formatierung der S-Sätze
sähe wie folgt aus: S-Sätze: 26; 36/37; 39.
Tabelle \ref{tab:s_phrases} zeigt die ausformulierten S-Sätze. Diese
inkonsistenten Daten müssen manuell überarbeitet werden.
\\
\begin{minipage}[h]
 {\textwidth}
 \centering %wenn tabelle mittig ausgerichtet sein soll
  \begin{tabular}{|p{0.2\textwidth}|p{0.6\textwidth}|l|l|}
   \hline
   \textbf{S-Satz} & \textbf{Anweisung}\\ \hline \hline 
     S 26 & Bei Berührung mit den Augen gründlich mit Wasser abspülen und Arzt
     konsultieren.
     \\ 
     \hline       
     S 36 & Bei der Arbeit geeignete Schutzkleidung tragen.
     \\ 
     \hline 
     S 37 & Geeignete Schutzhandschuhe tragen.
     \\ 
     \hline 
     S 39 & Schutzbrille/Gesichtsschutz tragen.
     \\  
     \hline 
     S 36/37 & Bei der Arbeit geeignete Schutzhandschuhe und Schutzkleidung
     tragen.
     \\ 
     \hline 
  \end{tabular}
 \captionof{table}{Mehrere S-Sätze und deren Sicherheitsanweisungen}
 \label{tab:s_phrases}
\end{minipage}
\\
\\
Diese Fehler sind nicht immer gleichförmig und können daher nicht automatisiert
erkannt werden. Das Auftreten von Fehlern ist willkürlich. Man kann keine
Systemathik identifizieren, was eine automatische Erkennung und Behebung weiter
erschwert.
\\
Die Fehler haben ihren Ursprung bei den Herstellern der Chemikalien. Einige
geben ihre Daten in Form einer \sona{Excel}-Tabelle an den Betreiber von
\sona{euSDB} weiter, damit dieser sie auf der Plattform veröffentlicht. Die
Kontrolle der Daten auf Richtigkeit erfolgt dabei händisch und nicht
automatisiert. Daher ist es immer wieder passiert, dass Datensätze in der
falschen Notation in das System eingegeben wurden.
\\
Die Fehler konnten entstehen, da praktisch alle Daten als Text in einer einzigen
Tabelle abgelegt wurden. Zwar gab es Vorgaben zur Notation von Daten, doch diese
für jedes Textfeld einzeln zu prüfen, ist aus Zeitgründen nicht immer möglich
gewesen.
\\
Zur Zeit stehen die \ac{SDB} über einen eigenen Link als \ac{PDF} zum
Herunterladen bereit. Schöner wäre es, wenn man die Möglichkeit erhält, sich
das \ac{SDB} über eine Parameterübergabe anzeigen zu lassen. Dabei
wird daran gedacht, dass man entweder das Format "`pdf"' an die Ressource
"`chemicals"' übergibt, oder über den \ac{HTTP}-Header diesen Aufruf auslöst.
Dieser Aufruf funktioniert nur bei der Ressource "`chemicals"'. Hier wird
dann das \ac{PDF} aufgerufen und dem Kunden zum Download bereitgestellt.
\\
Aufgrund des strikt eingehaltenen Entwicklungsstils eines modularen Aufbaus ist
es leicht, die Plattform zu erweitern. Dabei ist es
auch nicht zwingend notwendig, das bestehende System in allen Komponenten
verstanden zu haben. Versteht man einzelne Komponenten nicht, oder weiß man
diese nicht zu nutzen, kann man sie auch in der eigenen Erweiterung selbst neu
implementieren. Dies hat die Vorteile, dass man den eigenen Quellcode
verstanden hat und von den anderen Entwicklungen unabhängig bleibt.
\\
Folgeprojekte können sich auch aus Nicht-Serverentwicklungen ergeben. Z.B gibt
es ein Interesse an einem Client-Dienst, der automatisch Etiketten und
Beschriftungen für Aufbewahrungsbehälter von Chemikalien aus den \ac{SDB} erstellt. Das
manuelle Erstellen solcher Etiketten ist sehr zeitintensiv und somit auch
teuer.
Der Etiketten-Client soll dabei mittels der \ac{REST}-konformen \ac{API} auf die
Datenbank zugreifen können, die benötigten Daten abrufen und dann ein Etikett
zusammenstellen und zum Ausdrucken bereithalten.
\\
Ein weiterer Client der entwickelt werden soll, sieht wie folgt aus: Jedes
Unternehmen, das Chemikalien lagert, muss sehr genau buchführen, wo sich die
Chemikalien befinden. Ebenfalls muss es in dieser Übersicht alle
Gefährdungshinweise und Arbeitsanweisungen aufgelistet sein. Bei letzteren
Informationen kommt die Datenbank von \sona{euSDB} ins Spiel. Der zu
entwickelnde Client soll zum einen die vom Kunden eingetragenen Lagerungsorte
tabellarisch enthalten und diese Datensätze um die Informationen aus der
Datenbank erweitern.

\section{Zusammenfassung}
\label{sec:Zusammenfassung}

Die neue Plattform \sona{euSDB} stellt den gesamten Funktionsumfang der alten
Plattform zur Verfügung. Es existiert eine Suchfunktion, eine
Zugriffsverwaltung, ein Downloadbereich für \ac{SDB}, usw.. Die Umsetzung der
Plattform, also die Programmierung, orientiert sich streng an verschiedenen
Programmierparadigmen, wie z.B. Objektorientierung und lose Kopplung.
\\
Das Projekt kann auf allen Unix- und Windows-Versionen laufen. Es ist damit
relativ betriebssystemunabhängig. Dem Betreiber von \sona{euSDB} wird dadurch
freigestellt, auf welchem System er seinen Server laufen lässt.
\\
Dem Auftraggeber steht eine vollwertige Onlineplattform zur Verfügung, die an
nur wenigen Punkten einer Überarbeitung bedarf, sodass sie in den Online-Betrieb
gehen kann. Dabei geht es vor allem darum, irreguläre Abfragen abzufangen und zu
verhindern, dass das System abstürzt.
\\
Durch die Möglichkeiten der Administration, die nicht mehr ausschließlich an der
Datenbank über einen Datenbankmanager durchgeführt werden, wird für eine Bessere
Datenintegrität gesorgt. Das System merkt direkt, wenn Fremdschlüssel gelöscht
werden, obwohl sie noch gebraucht werden und verhindert in einem solchen Fall
das Löschen.
\\
Die Entwicklung der Plattform hat die Mächtigkeit und die vielen Möglichkeiten
des \sona{Django}-Framework aufgezeigt. Dabei bleibt sie jedoch nachvollziehbar
und leicht zu erlernen. Auch mit nicht so umfangreichen Vorkenntnissen in
Webentwicklung und \sona{Python} scheint es effektiv einsetzbar zu sein. Im Web
werden viele Erweiterungen und \ac{APP} zur Verfügung gestellt. Diese lassen
sich leicht nachvollziehen und einfach in das bestehende System einbinden.
\\
Einen Nachteil des Frameworks ist es, dass sich Fehler erzeugen lassen, die die
\sona{Python}-Umgebung abstürzen lassen. Solche Fehler müssen unbedingt behoben
werden. Erst dann kann der Server neugestartet werden. Dies kann zu
längeren Ausfallzeiten der Plattform führen, wenn man nicht sofort realisiert
hat, dass der Server außer Betrieb ist. Es ist sehr wichtig, das System
gründlich zu testen, damit möglichst keine Fehler auftreten.
\\
Der Auftraggeber erhält mit der neuentwickelten Softwareplattform \sona{euSDB}
eine ausgiebig getestete, ihn im Funktionsumfang nicht einschränkende und
stabile Lösung. Sie gibt ihm die Möglichkeit, sie fast beliebig um neue
Funktionen zu erweitern. Auch die Cliententwicklung bietet neue Möglichkeiten.
Auf Grund der strikt eingehaltenen losen Kopplung ist sie nicht an eine gewisse
Programmiersprach gebunden. \sona{euSDB} stellt insofern
eine höchst flexible Plattform dar.
\\
Einige Teile des Projektes sind bisher sehr aufwändig gelöst, da der
Programmierer mit \sona{euSDB} das erste Mal ein \sona{Django}-System realisiert
hat. Es fehlten ihm Kenntnisse und Erfahrungen in dieser Programmiersprache,
sodass einige Komponenten erst während der Realisierung verstanden und nachvollzogen wurden.
Die zu diesem Zeitpunkt bereits realisierten Funktionen können ineffizient sein
und einer Überarbeitungen bedürfen, z.B. die
\ac{REST}-konforme \ac{API} ist nicht in einer eigenen \ac{APP} untergebracht.
\\
Der Programierer und die Firma \fina{innoQ} freuen sich auf weitere Projekte,
die der Auftraggeber im Rahmen der Plattform \sona{euSDB} vergibt.
Abschließend kann man sagen, dass das Projekt zwar noch nicht vollständig
abgeschlossen ist, jedoch das bis dahin fertiggestellte sowohl aus Sicht des
Auftraggebers als auch aus Sicht von \fina{innoQ} als positiv gewertet wird und
somit auch von einem erfolgreichen Projektverlauf gesprochen werden kann.
