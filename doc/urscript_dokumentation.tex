\documentclass[
a4paper,
12pt,
oneside,
headsepline,		% Linie für Kopfzeile
footsepline,		% Linie für Fußzeile
%bibtotoc
]{scrbook}
 
% Druckbereich: \areaset[BCOR]{textwidth}{textheight}
% BCOR ist "Binding Correction", also wieviel Innenrand verloren geht
% A4 hat 297mm x 210mm
% wenn keine Marginalien, dann ist Breite 15cm vielleicht besser
\areaset[1.5cm]{14cm}{25cm}
 
%% Die folgende Zeile sorgt dafür, daß die Fußnoten eingerückt werden,
%% und zwar um 2em (class scrbook).
\deffootnote{2em}{2em}{\textsuperscript{\normalfont\thefootnotemark} }
 
\usepackage[utf8x]{inputenc}  % Unterstützung für Unicode-Zeichen-Eingabe
\usepackage[T1]{fontenc}      % Unterstützung für Europäische-Zeichen-Ausgabe
\usepackage{ae}               % verbesserte Unterstützung für Umlaute
\usepackage[german]{babel}    % deutsche Übersetzungen und Wortumbrüche
\usepackage[scaled]{helvet}  % schönere Schriftart: Helvetica
\usepackage{mathptmx}            % passende Mathe-Schriftart
\usepackage{courier}             % passende Monospaced-Schriftart
\usepackage{pgf}              % Unterstützung für Graphiken
\usepackage{tikz}             % Unterstützung für Graphiken
\usepackage[onehalfspacing]{setspace}
\usepackage{acronym} 
\usepackage{listings}
\usepackage{color}
\definecolor{Gray}{gray}{0.9}
\definecolor{sun1}{rgb}{0.2,0.2,0.4}
\definecolor{sun2}{rgb}{0.4,0.4,0.6}
\definecolor{sun3}{rgb}{0.6,0.6,0.8}
\definecolor{sun4}{rgb}{0.8,0.8,1}
\definecolor{msblau}{rgb}{0.31,0.4,0.517}
\definecolor{darkred}{rgb}{0.5,0,0}
\definecolor{darkgreen}{rgb}{0,0.5,0}
\definecolor{darkblue}{rgb}{0,0,0.5}
 
\usepackage[                
   pdftex,                  % Ausgabe-Medium: PDF
   colorlinks=true,         % farbige Links in der Bildschirm-Version?
   pdfstartview=FitV,       % wie soll Acrobat starten?
   linkcolor=blue,         % Farbe für Querverweise
   citecolor=blue,         % Farbe für Zitierungen
   urlcolor=blue,          % Farbe für Links
   bookmarks=true
   ]{hyperref}              % Paket für Links im PDF
 
%%%% Informationen über den Text festlegen %%%%%%%%%%%%%%%%%%%%%%%%%%%%%%%%%%
\title{Dokumenation des URScripts für die Beispielanwendung}
\author{Andreas Collmann}
\date{\today}
 
%%% hier können noch viel viel mehr Einstellungen kommen
%%%% hier beginnt der Inhalt %%%%%%%%%%%%%%%%%%%%%%%%%%%%%%%%%%%%%%%%%%%%%%%%
%\spacing{1.5}

\makeindex

\newcommand{\eruck}[1]{
    \noindent\hspace*{#1}
}
\newcommand{\zab}[1]{
    \vspace*{#1}
}
\newcommand{\fina}[1]{
    {\em #1}
}
\newcommand{\finar}[1]{
    {\fina{#1}\textsuperscript{\textregistered}}		
}
\newcommand{\sona}[1]{
    {\em #1}
}
\newcommand{\code}[1]{
    {\\ \eruck{20mm} \texttt{#1} \\}
}

% Formatierung Quellcode

\lstset{
    language=Python,
	numbers=left,
    breaklines=true,
	backgroundcolor=\color{Gray},
	basicstyle=\small,
	showspaces=false,
    basicstyle=\scriptsize\ttfamily,
    keywordstyle=\bfseries\ttfamily\color{orange},
    stringstyle=\color{blue}\ttfamily,
    commentstyle=\color{gray}\ttfamily,
    emph={square}, 
    emphstyle=\color{blue}\texttt,
    emph={[2]root,base},
    emphstyle={[2]\color{orange}\texttt},
    showstringspaces=false,
    flexiblecolumns=false,
    tabsize=2,
    numbers=left,
    numberstyle=\tiny,
    numberblanklines=false,
    stepnumber=1,
    numbersep=10pt,
    xleftmargin=15pt
	}

\begin{document}

\chapter{Beispielanwendung mit speichern der Daten}

\begin{lstlisting}
def myProg():

  global richtung=False

  // festlegen der maximalen beschleunigung über die a_max variable
  global a_max = d2r(40)

  // festlegen er maximalen geschwindigkeit über die v_max variable
  global v_max = d2r(60)

  // Startpunkt für blaue seite des würfels
  global blue_deg = [d2r(79.34), d2r(-130.29), d2r(-108.28), d2r(-31.43), d2r(-90.0), d2r(-1.02)]


  // Funktione rotiert den Würfel über die übergebene Aches (1= X-Achse, 2=Y-Achse, 3=Z-Achse) um a grad
  def rotate_axis(a, b):
    rotationPos=p[0.0, 0.0, 0.0, 0.0, 0.0, 0.0]
    a= d2r(a)
    if b == 1:
      rotationPos = p[0.0, 0.0, 0.0, a, 0.0, 0.0]
    elif b == 2:
      rotationPos = p[0.0, 0.0, 0.0, 0.0, a, 0.0]
    elif b == 3:
      rotationPos = p[0.0, 0.0, 0.0, 0.0, 0.0, a]
    end
    movel(pose_add(get_actual_tcp_pose(), rotationPos), a=0.5, v=0.1)
  end


  // Funktion dreht den letzten Joint bei digital in HIGH zur Nächsten form um 71.299 grad, Bis alle formen auf der Ebene erreicht wurden
  def move_toy():
    schleife_1 = 4
    while schleife_1 > 0:
      while digital_in[1] == False:
        sync()
      end
      r = d2r(71.299)
      pos = get_actual_joint_positions()
      pos= [pos[0], pos[1], pos[2], pos[3], pos[4], pos[5]+r]
      movej(pos, a=a_max, v=v_max)
      schleife_1=schleife_1-1
    end
  end

  // justiert den roboter in die richtung vorgegeben von der variable direction
  thread justify():
    if direction  ==  True:
      move_z=0.5
    else:
      move_z=-0.5
    end
    movel(pose_add(get_actual_tcp_pose(), p[0.0,0.0, move_z, 0.0,0.0,0.0]), a=0.1, v=0.15)
  end

  // Funktion Fragt Spieler ob der Robotre in der höhe Justiert werden soll
  def justify_height():
    // Ja/Nein anfrage vom Benutzer holen ob justiert wird
    global justieren = request_boolean_from_primary_client("roboter justieren?")
    varmsg("justieren", justieren)
    while (justieren ==  True ):
    // Ja/Nein anfrage vom Benutzer holen in welche richtung justiert wird
      global richtung = request_boolean_from_primary_client("hoch ~> ja ; runter ~> nein")
      varmsg("richtung", richtung)
      # wait until button 1 is pressed
      while digital_in[1] == False:
        sync()
      end
      // justify function bewegt den Roboter solange der digitale eingang auf high steht in die vorgegebene richtung
      justify_handler= run justify()
      while digital_in[1] == True:
        sync()
      end
      kill justify_handler
      blue_deg= get_actual_joint_positions()
      justieren = request_boolean_from_primary_client("roboter erneut justieren?")
      varmsg("justieren", justieren)
    end
  end

  // funktion fragt ob ein neuer spieler erstellt werden soll, wenn ja, wird über die socket verbindung der spieler mit dem neuen Namen gespeichert.
  def spieler_erst():
    neu = request_boolean_from_primary_client("Neuen Spieler erstellen?")
    name = request_string_from_primary_client("Name des Spielers?")
    if neu == True:
      socket_send_string("new_patient")
      socket_send_string(name)
    end
  end

  // Abfrage vom programm bei programmstart ob das programm weiterlaufen soll.
  start_var = request_boolean_from_primary_client("program starten?")
  varmsg("start_var", start_var)

  // öffnen der socket verbindung zum Daten Server
  data_server = socket_connect("141.100.101.48", 8000)
  
  
  if start_var == True:
    movej(blue_deg, a=a_max, v=v_max)
    if( data_server):
      spieler_erst()
    else:
      global p_name = request_string_from_primary_client("name des Spielers?")
      varmsg("p_name", p_name)
    end
    justify_height()

    // laufe solange gespielt werden soll
    while start_var == True:
      movej(blue_deg, a=a_max, v=v_max)
      popup("Wenn Wuerfel entleet ist, kann losgelegt werden(Mit Button 1 den Wuerfel drehen.)", "Nachricht", False, False)
      move_toy()
      movej(red_deg, a=a_max, v=v_max)
      rotate_axis(-40, 1)
      move_toy()
      start_var = request_boolean_from_primary_client("erneut beginnen?")
      varmsg("start_var", start_var)
    end
  end
  popup("programm wird beendet", "Nachricht", False, False)
  socket_close()
end
\end{lstlisting}

\end{document}