 \chapter{Glossar}
 \label{sec:Glossar_glo}

\begin{acronym}[]%% in [] längste zu erwartende Abkürzung
%\setlength{\itemsep}{-\parsep}
 
 \acro{Syntax Highligting}{}: Zur Verbesserung der Lesbarkeit und der Übersicht, wird in einem Textverarbeitungsprogramm der Programmcode unterschiedlich dargestellt. Meist mit unterschiedlichen Farbwerten. Der Entwickler sieht mit einem Blick ob er es mit Textvariablen, Zahlenwerten zu tun hat.
 \acro{ISO}{International Organization for Standardization}: Die ISO ist eine Internationale Vereinigung um Standartisierte Normen in der Industrie zu erarbeiten und festzulegen. Jedes Land, dass Mitglied ist, muss sich an diese Normen halten. Es gibt fast kein Land, dass nicht Mitglied ist.
 \acro{PTP}{Point to Point}: PTP in Deutsch auch Punktsteuerung genannt, ist die einfachste Methode um einen Roboter auf einen anderen Zielpunkt zu fahren.
 \acro{API}{Application programming interface}: Eine Schnittstelle um eine Software mit einer anderen Software zu verbinden. Die Schnittstelle in Form eines Programmteils wird öffentlich gemacht und gut Dokumentiert. Die Externe Software benutzt diesen Programmteil um die Software mit der Schnittstelle zu nutzen.
 \acro{URP}{Univeral Robot Program}: URP ist eine Dateiendung für ein Programm geschrieben über die Polyscope Software.
 \acro{UR}{Universal Robots}: UR ist eine Dänische Firma die den UR5 Roboter Herstellt.
 \acro{Software Bibliotheken}{}: Eine Software Bibliothel, oft auch Modul oder Library genannt, ist eine Kapselung von Programmcode der wiederverwendet werden kann.
 \acro{TCP/IP}{Transmission Control Protocol / Internet Protocol}: TCP/IP ist beinhalten mehrere Netzwerkprotokolle, die es ermöglichen, dass man mehrere Rechner Vernetzen und Nachrichten austauschen lassen.
 \acro{Popup}{}: Ein Fenster oder anderes Visuelles Element um einem Benutzer einer Anwendung Nachrichten zukommen zu lassen.
 \acro{parsen}{}: Parser: Informationen zerlegen und entsprechend interpretieren.
 \acro{Mock}{}: Ein Platzhalter für Software Objekte. Wird benutzt um Software zu testen, bei dem ein Teil der Software noch nicht existiert oder ausgeschlossen werden soll.
 \acro{Interpreter}{}: In der Softwareentwicklung sind Interpreter die Kernpunkte von Programmiersprachen die den Code nicht in Machinensprache Kompilieren. Interpreter lesen den Textcode analysieren ihn auf fehler und führen ihn nach der Analyse aus.
 \acro{Big-Endian Format}{}: 
 \acro{Little-Endian Format}{}: 
 % \acro{}{}:
\end{acronym}
