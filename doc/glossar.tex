 \chapter{Glossar}
 \label{sec:Glossar_glo}

\begin{acronym}[Syntax Highligting]%% in [] längste zu erwartende Abkürzung
%\setlength{\itemsep}{-\parsep}
 
 \acro{ISO}{International Organization for Standardization}: Die ISO ist eine Internationale Vereinigung um Standartisierte Normen in der Industrie zu erarbeiten und festzulegen. Jedes Land, dass Mitglied ist, muss sich an diese Normen halten. Es gibt fast kein Land, dass nicht Mitglied ist.
 \acro{PTP}{Point to Point}: PTP in Deutsch auch Punktsteuerung genannt, ist die einfachste Methode um einen Roboter auf einen anderen Zielpunkt zu fahren.
 \acro{API}{Application Programming Interface}: Eine Schnittstelle um eine Software mit einer anderen Software zu verbinden. Die Schnittstelle in Form eines Programmteils wird öffentlich gemacht und gut Dokumentiert. Die Externe Software benutzt diesen Programmteil um die Software mit der Schnittstelle zu nutzen.
 \acro{URP}{Univeral Robot Program}: URP ist eine Dateiendung für ein Programm geschrieben über die Polyscope Software.
 \acro{UR}{Universal Robots}: UR ist eine Dänische Firma die den UR5 Roboter Herstellt.
 \acro{TCP/IP}{Transmission Control Protocol / Internet Protocol}: TCP/IP ist beinhalten mehrere Netzwerkprotokolle, die es ermöglichen, dass man mehrere Rechner Vernetzen und Nachrichten austauschen lassen.
 \acro{Interpreter}{} In der Softwareentwicklung sind Interpreter die Kernpunkte von Programmiersprachen die den Code nicht in Machinensprache kompilieren. Interpreter lesen den Textcode, analysieren ihn auf Fehler und führen ihn nach der Analyse aus.
 \acro{Big-Endian}{Big-Endian-Format}: Big-Engian-Format ist die Festlegung der Byte-Reihenfolge, wie das Computersystem Speicherbereiche interpretieren und beschreiben soll. Dieses Format legt fest, dass das höchstwertigste Bit an der kleinsten Speicheradresse liegt.
 \acro{Little-Endian}{Little-Endian-Format}: Wie bei \acs{Big-Endian}, legt das Little-Endian-Format die Byte-Reihenfolge fest. Mit Little-Endian jedoch wird das niedrigwertige Bit an die kleinste Speicheraddresse gesetzt.
 \acro{TCP}{Tool Center Point}: Der TCP beschreibt den Mittelpunkt des Werkzeugs, der in der Regel an der Spitze des Roboters angebracht ist.
 \acro{teachen}{}: In der Robotik bedeutet teachen das Anlernen von Wegpunkten, durch handliches Führen des Roboter.
 \acro{SCP}{Secure Copy}: Ein Linux Programm zum sicheren Austausch von Daten zwischen zwei verschiedenen Host-Systemen
 \acro{GUI}{Graphical User Interface}: Ein GUI erlaubt es einem Benutzer mit einem Programm über graphische Symbole zu interagieren
 % \acro{}{}:
\end{acronym}
