\documentclass[
a4paper,
12pt,
oneside,
headsepline,		% Linie für Kopfzeile
footsepline,		% Linie für Fußzeile
%bibtotoc
]{scrbook}
 
% Druckbereich: \areaset[BCOR]{textwidth}{textheight}
% BCOR ist "Binding Correction", also wieviel Innenrand verloren geht
% A4 hat 297mm x 210mm
% wenn keine Marginalien, dann ist Breite 15cm vielleicht besser
\areaset[1.5cm]{14cm}{25cm}
 
%% Die folgende Zeile sorgt dafür, daß die Fußnoten eingerückt werden,
%% und zwar um 2em (class scrbook).
\deffootnote{2em}{2em}{\textsuperscript{\normalfont\thefootnotemark} }
 
\usepackage[utf8x]{inputenc}  % Unterstützung für Unicode-Zeichen-Eingabe
\usepackage[T1]{fontenc}      % Unterstützung für Europäische-Zeichen-Ausgabe
\usepackage{ae}               % verbesserte Unterstützung für Umlaute
\usepackage[german]{babel}    % deutsche Übersetzungen und Wortumbrüche
\usepackage[scaled]{helvet}  % schönere Schriftart: Helvetica
\usepackage{mathptmx}            % passende Mathe-Schriftart
\usepackage{courier}             % passende Monospaced-Schriftart
\usepackage{pgf}              % Unterstützung für Graphiken
\usepackage{tikz}             % Unterstützung für Graphiken
\usepackage[onehalfspacing]{setspace}
\usepackage{acronym} 
\usepackage{listings}
\usepackage{color}
\definecolor{Gray}{gray}{0.9}
\definecolor{sun1}{rgb}{0.2,0.2,0.4}
\definecolor{sun2}{rgb}{0.4,0.4,0.6}
\definecolor{sun3}{rgb}{0.6,0.6,0.8}
\definecolor{sun4}{rgb}{0.8,0.8,1}
\definecolor{msblau}{rgb}{0.31,0.4,0.517}
\definecolor{darkred}{rgb}{0.5,0,0}
\definecolor{darkgreen}{rgb}{0,0.5,0}
\definecolor{darkblue}{rgb}{0,0,0.5}
 
\usepackage[                
   pdftex,                  % Ausgabe-Medium: PDF
   colorlinks=true,         % farbige Links in der Bildschirm-Version?
   pdfstartview=FitV,       % wie soll Acrobat starten?
   linkcolor=blue,         % Farbe für Querverweise
   citecolor=blue,         % Farbe für Zitierungen
   urlcolor=blue,          % Farbe für Links
   bookmarks=true
   ]{hyperref}              % Paket für Links im PDF
 
%%%% Informationen über den Text festlegen %%%%%%%%%%%%%%%%%%%%%%%%%%%%%%%%%%
\title{Dokumenation der C-API mit Beispielanwendung}
\author{Andreas Collmann}
\date{\today}
 
%%% hier können noch viel viel mehr Einstellungen kommen
%%%% hier beginnt der Inhalt %%%%%%%%%%%%%%%%%%%%%%%%%%%%%%%%%%%%%%%%%%%%%%%%
%\spacing{1.5}

\makeindex

\newcommand{\eruck}[1]{
    \noindent\hspace*{#1}
}
\newcommand{\zab}[1]{
    \vspace*{#1}
}
\newcommand{\fina}[1]{
    {\em #1}
}
\newcommand{\finar}[1]{
    {\fina{#1}\textsuperscript{\textregistered}}		
}
\newcommand{\sona}[1]{
    {\em #1}
}
\newcommand{\code}[1]{
    {\\ \eruck{20mm} \texttt{#1} \\}
}

% Formatierung Quellcode

\lstset{
    language=Python,
	numbers=left,
	backgroundcolor=\color{Gray},
	basicstyle=\small,
	showspaces=false,
    basicstyle=\scriptsize\ttfamily,
    keywordstyle=\bfseries\ttfamily\color{orange},
    stringstyle=\color{blue}\ttfamily,
    commentstyle=\color{gray}\ttfamily,
    emph={square}, 
    emphstyle=\color{blue}\texttt,
    emph={[2]root,base},
    emphstyle={[2]\color{orange}\texttt},
    showstringspaces=false,
    flexiblecolumns=false,
    tabsize=2,
    numbers=left,
    numberstyle=\tiny,
    numberblanklines=false,
    stepnumber=1,
    numbersep=10pt,
    xleftmargin=15pt
	}

\begin{document}

\chapter{Vorbereitung}

\section{benötigte Bibliotheken}

Universal Robots stellt mehrere Bibliotheken zur verfügung. 

\begin{itemize}
\item libcollision.a
\item libconfiguration.a
\item libdev.a
\item libkinematics.a
\item libmath.a
\item librobotcomm.a
\end{itemize}

Die Lib ``librobotcomm.a'' bietet die Funktionen, die nötig sind um mit dem Controller zu kommunizieren. Die dazugehörige Header Datei ist 
``robotinterface.h''.

\subsection{Funktionen der Lib librobotcomm.a}

\begin{lstlisting}


/* robot modes */
#define ROBOT_RUNNING_MODE 0
#define ROBOT_FREEDRIVE_MODE 1
#define ROBOT_READY_MODE 2
#define ROBOT_INITIALIZING_MODE 3
#define ROBOT_SECURITY_STOPPED_MODE 4
#define ROBOT_EMERGENCY_STOPPED_MODE 5
#define ROBOT_FATAL_ERROR_MODE 6
#define ROBOT_NO_POWER_MODE 7
#define ROBOT_NOT_CONNECTED_MODE 8
#define ROBOT_SHUTDOWN_MODE 9
#define ROBOT_SAFEGUARD_STOP_MODE 10

#ifdef __cplusplus
extern "C" {
#endif

  uint64_t robotinterface_get_step();
  double robotinterface_get_time();
  double robotinterface_get_time_per_step();

  /**
   * \name Robot commands
   * This is where the robot command methods start.
   * @{
   */
  void robotinterface_command_position_velocity_acceleration(const double *q,
                                                             const double *qd,
                                                             const double
                                                       *qdd);

  /**
   * \name Robot commands
   * This is where the robot command methods start.
   * @{
   */
  void robotinterface_command_velocity_acceleration(const double *q,
                                                             const double *qd,
                                                             const double
                                                             *qdd);
  void robotinterface_command_velocity(const double *qd);
  void robotinterface_command_joint_position_velocity_acceleration(int joint,
                                                                   double q,
                                                                   double qd,
                                                                   double
                                                                   qdd);

  void robotinterface_command_velocity_security_torque_control_torque(const double *qd,
                                                             const double *security_torque,
                                                             const double *control_torque,
                                                             const double *softness
                                                             );

  void robotinterface_command_empty_command();

  void robotinterface_master_goto_bootloader_mode();

  /* @{ */
  /** Value must be 0 or 1, port can be 0..9 */
  void robotinterface_command_digital_out_port(int port, int value);
  void robotinterface_command_digital_out_bits(unsigned short bits);
  /** Value must be 0 or 1, port can be 0..9 */
  void robotinterface_command_analog_out_port(int port, double value);

  /**
   * Value must be 0 or 1, port can be 0..9
   * For port 0 and 1, the controller box analog inputs;
   * - Range must be 0,1,2,3
   * For port 2 and 3, the tool board analog inputs;
   * - Range 0 is 0..5 Volt
   * - Range 1 is 0..10 Volts
   * - Range 4 is 4..20 mA
   */
  void robotinterface_command_analog_input_range_port(int port, int range);
  void robotinterface_command_analog_output_domain(int port, int type);
  void robotinterface_command_tool_output_voltage(unsigned char value);

  /* The TCP is regarded as part of the robot
   * Get and set methods for TCP and TCP payload are as follows
   */
  void robotinterface_set_tcp(const double *tcp_pose);
  void robotinterface_set_tcp_payload(double tcp_payload);
  void robotinterface_get_tcp(double *tcp_pose);
  double robotinterface_get_tcp_payload();
  /* @} */
  /* @} */

  /* Funtions to set/get force and torque (wrench) at the TCP */
  void robotinterface_set_tcp_wrench(const double *new_tcp_wrench, const int in_base_coord);
  void robotinterface_get_tcp_wrench(double *gotten_tcp_wrench);
  int robotinterface_is_tcp_wrench_in_base_coord();


  /**
   * \name Communication
   * This is where the robot communcation methods start.
   *
   * \{
   * \brief Initialize connection to the robot.
   *
   * If robot is directly connected, it returns TRUE,
   * otherwise it will return false and enter simulation mode.
   *
   * simulation mode can also be simulated by setting argument
   * simulated = 1.
   */
  int robotinterface_open(int open_simulated);
  int robotinterface_do_open(int open_simulated);
  int robotinterface_open_allow_bootloader(int open_simulated);
  void robotinterface_read_state_blocking();
  void robotinterface_send();

  /* ! physical robot connected? (Ethernet talking to coldfire) */
  int robotinterface_is_robot_connected();


  /* ! Shuts down the communication channel to the robot. */
  int robotinterface_close();


  /**
   * Returns robot mode, which is one of the following:
   *
   * - ROBOT_RUNNING_MODE
   * - ROBOT_READY_MODE
   * - ROBOT_INITIALIZING_MODE
   * - ROBOT_STOPPED_MODE
   * - ROBOT_SECURITY_STOPPED_MODE
   * - ROBOT_FATAL_ERROR_MODE
   * - ROBOT_NOT_CONNECTED_MODE
   */
  uint8_t robotinterface_get_robot_mode();

  /**
   * Sets the robot in ROBOT_READY_MODE if current mode is ROBOT_RUNNING_MODE
   */
  void robotinterface_set_robot_ready_mode();

  /**
   * Sets the robot in ROBOT_RUNNING_MODE if current mode is
   * ROBOT_READY_MODE or ROBOT_FREEDRIVE_MODE
   *
   */
  void robotinterface_set_robot_running_mode();

  /**
   * Sets the robot in ROBOT_FREEDRIVE_MODE if current mode is
   *  ROBOT_RUNNING_MODE
   *
   */
  void robotinterface_set_robot_freedrive_mode();


  /**
   * Returns the state of one joint.
   *
   * Each joint can be in one of the
   * following states.
   *
   * - JOINT_MOTOR_INITIALISATION_MODE
   * - JOINT_BOOTING_MODE
   * - JOINT_DEAD_COMMUTATION_MODE
   * - JOINT_BOOTLOADER_MODE
   * - JOINT_CALIBRATION_MODE
   * - JOINT_STOPPED_MODE
   * - JOINT_FAULT_MODE
   * - JOINT_RUNNING_MODE
   * - JOINT_INITIALISATION_MODE
   * - JOINT_IDLE_MODE
   */
  uint8_t robotinterface_get_joint_mode(int joint);

  /**
   * Returns the state of the tool
   *  Either:
   * - JOINT_IDLE_MODE
   * - JOINT_BOOTLOADER_MODE
   * - JOINT_RUNNING_MODE
   */
  uint8_t robotinterface_get_tool_mode();

  /**
   * Returns number_of_message that needs to be read by
   * #robotinterface_get_next_message().
   */
  uint8_t robotinterface_get_message_count();

  /* Revisit this list when making a new logger system */
  /* Source is:
   69: Safety sys 2
   68: Euromap67 2
   67: Euromap67 1
   66: Teach Pendant 2
   65: Teach Pendant 1
   6: Tool
   5: Wrist 3
   4: Wrist 2
   3: Wrist 1
   2: Elbow
   1: Shoulder
   0: Base
  -1: Safetysys (errors reported by masterboard code
  -2: Controller
  -3: RTMachine
  -4: Simulated Robot
  -5: GUI
  -6: ControlBox (not used, was used in a Version mismatch check in the controller, the source for that message has been replaced with -2)
  */

  struct message_t {
    uint64_t timestamp;
    char source;
    char *text;
  };

  /**
   * Returns length of message copied to the char*
   */
  int robotinterface_get_message(struct message_t *message);

  /**
   * Takes a message as error codes rathern than text
   */
  int robotinterface_get_message_codes(struct message_t *msg, int *error_code,
                                       int *error_argument);

  /*@}*/

  /**
   * \name Security stops
   *
   * This is where the robot security stop methods start.
   * <em>Implemented in sequritycheck.c</em>
   *@{
   */

  /**
   * \return 0 if there is no error in the robot, true otherwise.
   */
  void robotinterface_power_on_robot();
  void robotinterface_power_off_robot();
  void robotinterface_security_stop(char joint_code, int error_state,
                                    int error_argument);
  int robotinterface_is_power_on_robot();
  int robotinterface_is_security_stopped();
  int robotinterface_is_emergency_stopped();
  int robotinterface_is_extra_button_pressed(); /* The button on the back side of the screen */
  int robotinterface_is_power_button_pressed();  /* The big green button on the controller box */
  int robotinterface_is_safety_signal_such_that_we_should_stop(); /* This is from the safety stop interface */
  int robotinterface_unlock_security_stop();
  int robotinterface_has_security_message();
  int robotinterface_get_security_message(struct message_t *message,
                                          int *error_state,
                                          int *error_argument);
  uint32_t robotinterface_get_master_control_bits();

  /*@}*/

  /**
   * \name Robot methods
   *
   * This is where actual robot methods start.
   *
   * Each function copies the resulting joint values of the
   * robot to the supplied array.
   *@{
   */
  void robotinterface_get_actual(double *q, double *qd);
  void robotinterface_get_actual_position(double *q);
  void robotinterface_get_actual_velocity(double *qd);
  void robotinterface_get_actual_current(double *I);
  void robotinterface_get_actual_accelerometers(double *ax, double *ay,
                                                double *az);
  double robotinterface_get_tcp_force_scalar();
  void robotinterface_get_tcp_force(double *F);
  void robotinterface_get_tcp_speed(double *V);
  double robotinterface_get_tcp_power();
  double robotinterface_get_power();

  /*@}*/

  int robotinterface_get_actual_digital_input(int port);
  int robotinterface_get_actual_digital_output(int port);
  unsigned short robotinterface_get_actual_digital_input_bits();
  unsigned short robotinterface_get_actual_digital_output_bits();
  double robotinterface_get_actual_analog_input(int port);
  double robotinterface_get_actual_analog_output(int port);
  unsigned char robotinterface_get_actual_analog_input_range(int port);
  unsigned char robotinterface_get_actual_analog_output_domain(int port);

  /** \name Ideal methods
   * This is where target (ideal) methods start.
   */
  /*@{*/
  void robotinterface_get_target(double *q, double *qd, double *qdd);
  void robotinterface_get_target_position(double *q);
  void robotinterface_get_target_velocity(double *qd);
  void robotinterface_get_target_acceleration(double *qdd);
  void robotinterface_get_target_current(double *I);
  void robotinterface_get_target_moment(double *m);

  unsigned short robotinterface_get_target_digital_output_bits();
  double robotinterface_get_target_analog_output(int port);
  unsigned char robotinterface_get_target_analog_input_range(int port);
  unsigned char robotinterface_get_target_analog_output_domain(int port);

  /*@}*/


  /**
   *
   * \name Temperature, Voltage and Current Methods
   */
  /*@{*/

  /* master */
  float robotinterface_get_master_temperature();  /* new */
  float robotinterface_get_robot_voltage_48V(); /* new */
  float robotinterface_get_robot_current(); /* new */
  float robotinterface_get_master_io_current(); /* new */
  unsigned char robotinterface_get_master_safety_state();
  unsigned char robotinterface_get_master_on_off_state();
  void robotinterface_safely_remove_euromap67_hardware();
  void robotinterface_disable_teach_pendant_safety();

  /* joint */
  void robotinterface_get_motor_temperature(float *T);
  void robotinterface_get_micro_temperature(float *T);
  void robotinterface_get_joint_voltage(float *V);

  /* tool */
  float robotinterface_get_tool_temperature();
  float robotinterface_get_tool_voltage_48V();
  unsigned char robotinterface_get_tool_output_voltage();
  float robotinterface_get_tool_current();

  /* Euromap for CB2.0*/
  uint8_t robotinterface_is_euromap_hardware_installed();
  uint8_t robotinterface_get_euromap_input(uint8_t input_bit_number);
  uint8_t robotinterface_get_euromap_output(uint8_t output_bit_number);
  void robotinterface_set_euromap_output(uint8_t output_bit_number, uint8_t desired_value);
  uint32_t robotinterface_get_euromap_input_bits();
  uint32_t robotinterface_get_euromap_output_bits();
  uint16_t robotinterface_get_euromap_24V_voltage();
  uint16_t robotinterface_get_euromap_24V_current();

  /* General purpose registers */
  uint16_t robotinterface_get_general_purpose_register(int address);
  void robotinterface_set_general_purpose_register(int address, uint16_t value);

  /*@}*/
\end{lstlisting}

\section{Compilieren}

Universal Robots bietet eine Beispielapplikation an. Diese Wird mit Scons Compiliert. Scons ist ein Python Programm
Folgend der Code zum Compilieren der main Dabei und das Linken dre Libs

\begin{lstlisting}
# Import construction environment
Import('env')
# cppath and libpath has to be adjusted to match the library directory of librobotcomm
cpppath = ['../libs']
libpath = ['../libs']
libs = ['robotcomm', 'kinematics', 'configuration', 'dev', 'collision', 'm', 'math', 'pthread']
ccflags = ['-O2', '-Wall', '-g']
env = Environment(CC = 'g++',
      CPPPATH = cpppath,
      LIBPATH = libpath,
      LIBS = libs,
      CCFLAGS = ccflags,
      CPPDEFINES = 'NDEBUG')

base_src = ['../libs/base_utils.c', '../libs/startup_utils.c', '../libs/interrupt_utils.c', '../libs/ur5lib.c', '../libs/helper.c', '../libs/tcphelper.c', '../libs/ur_kin.c']
tcp_src = ['main.c'] + base_src

tcp_prog = env.Program(target='ur5-server', source=tcp_src)

env.Default(tcp_prog)

###############################################################################
\end{lstlisting}

\chapter{Berechnung einer PTP Bahn}

Die Formeln für die Berechnung sind aus der Lektüre ``Industrieroboter: Methoden der Steuerung und Regelung '' von Wolfgang Weber 2. Ausgabe
herrausgebracht von  Carl Hanser Verlag GmbH & Co. KG

\begin{lstlisting}

// q is actual position of the six joints
// qz is the target position of the six joints
// move_pva struct will contain the final calculations for the interpolationphase
// profil  determines if the acceleration profil is a sinoidenprofil or eiter a rampenprofil

void calc_ptp_profile(const double * q, double * pz, MovePTPPacket *move_pva, int profil){
    //    double T[16];
    double factor= (profil == RAMPENPROFIL) ?  1.0 : 2.0;
    if(debug)
        printf("factor %f, profil %s\n", factor, profil == RAMPENPROFIL ? "Rampenprofil" : "Sinoidenprofil");
    double se= 0.0;
    int i, l=0;

    for(i=0; i < 6; i++){

// Berechnung der Benötigten Strecke anhand der vorgegebenen Geschwindigkeits und Beschleunigungswerte für jeden einzelnen Joint und ausrechnung des Leitjoints um eine Synchrone Bahn zu berechnen.
  
        move_pva->qst[i]=q[i];
        move_pva->qz[i] =pz[i];
        move_pva->v_max[i]= deg_to_rad(VELOCITY_RAD_MAX);
        move_pva->a_max[i]= deg_to_rad(ACCELERATION_RAD_MAX);
        se = pz[i] - q[i];
        move_pva->signse[i]= se < 0.0 ? -1.0 : (se > 0  ? 1.0 : 0.0);
        move_pva->se[i] = fabs(se);

// Falls die Strecke so minimal ist, das davon ausgegangen wird, 
// dass der joint schon an der richtigen Position steht, wird für diesen Joint keine Bahn berechnet
        if(move_pva->se[i] < deg_to_rad(EPS)){
            move_pva->tb[i]=0.0;
            move_pva->te[i]=0.0;
            move_pva->tv[i]=0.0;
            move_pva->v_max[i]=0.0;
            move_pva->a_max[i]=0.0;
        }else{

// anpassung der Geschwindigkeit, falls die Strecke zu kurz ist um auf die eigentliche Geschwindigkeit zu kommen.
            if(move_pva->v_max[i] > sqrt((move_pva->se[i]*move_pva->a_max[i])/factor))
                move_pva->v_max[i] = sqrt((move_pva->se[i]*move_pva->a_max[i])/factor);

// Berechnung anhand der Beschleunigungs und bremsphasen der einzelnen joints
            move_pva->tb[i] = (factor*move_pva->v_max[i])/move_pva->a_max[i];
            move_pva->te[i] = (move_pva->se[i] / move_pva->v_max[i]) + move_pva->tb[i];
            move_pva->tv[i] = move_pva->te[i] - move_pva->tb[i];
        }

// l ist der leitjoint, derjenige der am meisten zeit für seine strecke benötigt
        if(move_pva->te[i] > move_pva->te[l]){
            l=i;
        }
    }
// Falls Joint berechnet wird, anpassen der einznene Phasen an den Iterpolationstakt
    if(move_pva->se[l] < deg_to_rad(EPS)){
        // adapt longest joint to interpolationphase
        move_pva->tb[l] = (floor((factor*move_pva->v_max[l])/(move_pva->a_max[l]*T_IPO))+1)*T_IPO;
        move_pva->tv[l] = (floor(move_pva->se[l]/(move_pva->v_max[l]*T_IPO))+1)*T_IPO;
        move_pva->te[l] = move_pva->tv[l]+move_pva->tb[l];
        move_pva->v_max[l] = move_pva->se[l]/move_pva->tv[l];
        move_pva->a_max[l] = (factor*move_pva->se[l])/(move_pva->tv[l]*move_pva->tb[l]);
    }

//     adapt te of all joints to te of longest joint
    for(i=0;i<6;i++){
        if(l != i && move_pva->se[i] > deg_to_rad(EPS)){
            move_pva->te[i] = move_pva->te[l];
            if(profil == RAMPENPROFIL){
                move_pva->v_max[i] = (move_pva->a_max[i]*move_pva->te[l])/2 - sqrt(((POW(move_pva->a_max[i])*POW(move_pva->te[l]))/4)-(move_pva->se[i]*move_pva->a_max[i]));
            }else{
                move_pva->v_max[i] = (move_pva->a_max[i]*move_pva->te[l])/4 - sqrt(((POW(move_pva->a_max[i])*POW(move_pva->te[l]))-(8*move_pva->se[i]*move_pva->a_max[i]))/16);
            }

            if(move_pva->v_max[i] > sqrt((move_pva->se[i]*move_pva->a_max[i])/factor))
                move_pva->v_max[i] = sqrt((move_pva->se[i]*move_pva->a_max[i])/factor);

            move_pva->tv[i] = floor(move_pva->se[i]/(move_pva->v_max[i]*T_IPO))*T_IPO;
            move_pva->tb[i] = move_pva->te[i] - move_pva->tv[i];
            move_pva->v_max[i] = move_pva->se[i] / move_pva->tv[i];
            move_pva->a_max[i] = (factor*move_pva->v_max[i]) /  move_pva->tb[i];
        }
    }

    move_pva->interpolations= move_pva->se[l] < deg_to_rad(EPS) ? 0 :
                                                                 (int) round(move_pva->te[l] / T_IPO);


    if(debug) print_ptp_debug(move_pva, profil);

    return;
}

\chapter{Beispielprogramme}

\section{}

\end{document}