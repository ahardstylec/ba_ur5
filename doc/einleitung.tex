\chapter{Einleitung}
\label{einleitung}

\section{Fachliche Umgebung}
\label{fachliche_domaene}

Hauptaugenmerk dieser Arbeit ist es die möglichkeiten der Roboter-Mensch-Kollaboration
in der Industire, Medizin, Schule mit diesem Roboter zu prüfen.
Inwiefern der Mensch mit einem Heutigen Roboter mit entsprechenden Richtlinen programmiert werden kann um in den entsprechenden Feldern mit dem Menschen zusammen zu Arbeiten.

\section{Motivation und Ziel des Projektes}
\label{projektziel_motivation}

In der Industrie werden Roboter in den Fertigungsanlagen eingesetzt. 
Dies geschiet meist in Koordination mit anderen Robotern. In der nähe dieser Roboter, darf sich kein Mensch aufhalten, die Roboter sind umhaust, sprich in einem speziellen Bereich abgesichert, damit keine unfälle passieren können. 
Auf diese Weise kann man sehr effizient über automatisierte Fließbandstraßen Produkte herstellen.

Wenn jedoch eine sehr filigranere Arbeit gefragt ist, muss das Werkstück von einem Menschen bearbeitet werden, da der Mensch wesentlich bessere Fähigkeiten hat, auf Probleme zu reagieren oder Korrekturen vorzunehmen. In diesem fall wird die Fließbandstraße unterbrochen. Das Produkt muss aus dem umhausten Bereich gebracht werden, wo es von einem Menschen bearbeitet werden kann.

Für die Prodution wäre es viel sinnvoller und zeitsparender, wenn Roboter für den Menschen so sicher sind, dass keine Trennung zwichen Mensch und Robotern existiert.

In dem Bereich Pflege und der Medizin, müssen oft hebe arbeiten ausgeführt werden. Dies fürt dazu, dass die Menschen in solchen Berufen im späteren Altag mit Rückenproblemen oder ähnliche leiden leben müssen. Roboter, die eingesetzt werden um diese lasten abzunehmen, würde die Arbeit erleichtern und verletzungen vorbeugen.

Es soll untersucht werden inwiefern die Zusammenarbeit von Robotern und Menschen mit einem Roboter der die Sicherheitsauflagen erfüllt, realisiert werden kann.

\section{Aufgabenstellung}
\label{aufgabenstellung}

Es soll ein Anwendungsprogramm für alle möglichen Programmierschnittstellen für den Ur5 Roboter von Universal Robots entwickelt werden.
Dieses Anwendungsprogramm soll so ausgelegt sein, dass es als eine Beispielanwendung einer Roboter-Mensch Kollaboration ist.
Diese verschiedenen Programme werden miteinander verglichen. Es soll eine Entscheidungshilfe gegeben werden, für welchen Anwendungsfall, welche Schnittstelle am besten geeignet ist.

Die Programmierschnittellen sollen möglichst gut dokumentiert werden.

\section{Einordnung in die Themenfelder der Informatik}
\label{sec:einordnung}

Die Schnittstellen werden mit den Standart Programmiersprachen C/C++ und Python programmiert. Hinzu kommt noch von die eigens von Universal Robots entwickelte URScript sprache.
Da versucht wird den Roboter von einem anderen Rechner zu steuern, wird auch Netzwerkprogrammierung benötigt. Es muss eine kleines Protokoll entwickelt werden, mit dem der Roboter kommunizieren kann um Anwenderdaten zu speichern. 
