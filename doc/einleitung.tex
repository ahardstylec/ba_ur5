\chapter{Einleitung}
\label{einleitung}

Lorem ipsum dolor sit amet, consetetur sadipscing elitr, sed diam nonumy eirmod tempor invidunt ut labore et dolore magna aliquyam erat, sed diam voluptua. At vero eos et accusam et justo duo dolores et ea rebum. Stet clita kasd gubergren, no sea takimata sanctus est Lorem ipsum dolor sit amet. Lorem ipsum dolor sit amet, consetetur sadipscing elitr, sed diam nonumy eirmod tempor invidunt ut labore et dolore magna aliquyam erat, sed diam voluptua. At vero eos et accusam et justo duo dolores et ea rebum. Stet clita kasd gubergren, no sea takimata sanctus est Lorem ipsum dolor sit amet.

\section{Fachliche Umgebung}
\label{fachliche_domaene}

Die Fachliche Umgebung kann weit ausgedehnt werden. Hauptaugenmerk dieser Arbeit ist es die möglichkeiten der Roboter-Mensch-Kollaboration
in der Industire, Medizin, Schule mit diesem Roboter zu erkennen.
Inwiefern der Mensch mit einem Heutigen Roboter mit entsprechenden Richtlinen programmiert werden kann um in den entsprechenden Feldern mit dem Menschen zusammen zu Arbeiten.

\section{Motivation und Ziel des Projektes}
\label{projektziel_motivation}

In der Industrie werden schon Roboter in den Fertigungsanlagen eingesetzt. 
Dies geschiet meist in Koordination mit anderen Robotern. In der nähe dieser Roboter, darf sich kein Mensch aufhalten. 
Sie sind umhaust, sprich in einem speziellen Bereich abgesichert, damit keine unfälle passieren können. 
Auf diese weise kann man sehr gut automatische Fließbandarbeiten erledigen.

Wenn jedoch eine sehr filigranere Arbeit gefragt ist, muss das Werkstück von einem Menschen bearbeitet werden, da der Mensch wesentlich bessere Möglichkeiten hat auf probleme zu reagieren oder korrekturen vorzunehmen.

\section{Aufgabenstellung}
\label{aufgabenstellung}

Es soll ein Anwendungsprogramm für alle möglichen Programmierschnittstellen für den Ur5 Roboter von Universal Robots entwickelt werden.
Dieses Anwendungsprogramm soll so ausgelegt sein, dass es als eine Beispielanwendung einer Roboter-Mensch Kollaboration ist.
Diese verschiedenen Programme werden miteinander verglichen. Es soll eine Entscheidungshilfe gegeben werden, für welchen Anwendungsfall, welche Schnittstelle am besten geeignet ist.

Die Programmierschnittellen sollen möglichst gut dokumentiert werden.

\section{Einordnung in die Themenfelder der Informatik}
\label{sec:einordnung}

Es werden auf verschiedene Bereiche in der Informatik eingegangen. 
Es werden Themen angesprochen die im Bereich der Robotik liegen. 

