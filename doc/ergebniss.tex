\chapter{Ergebnis}
\label{sec:ErreichteErgebnisse}

Im Folgenden Kapitel werden die Schnittstellen werden gegenübergestellt und verglichen. Desweiterem werden die nicht erreichten Ziele erörtert.
\\\\
\begin{tabular}{|l|p{0.333\textwidth}|p{0.333\textwidth}|}
	\hline
	\textbf{Kriterium} & \textbf{C-API} & \textbf{Polyscope} 
	\\ \hline \hline 
	Programmierbarkeit & Schwer einfache Roboterprogramme zu Realisieren. & Leichter Einstieg zum Programmieren für Anfänger.
	\\ \hline
	Benutzerinteraktion &  Es ist möglich ein Übersichtliches \& intuitives Interface zu entwickeln & 
	Keine Komplexen Menüs möglich. Nur schwache Interaktion möglich.
	\\ \hline
	Testen & Test sind möglich, aber nur mit Simulation des Roboters & Keine eigenen Tests möglich. Getestet wird immer Live an Roboter.
	\\ \hline
	Debuggen & Compiler findet Syntax Fehler \& Debugging ist nur mit Simuliertem Roboter möglich & Bedingt möglich beim Testen Live am Roboter.
	\\ \hline
	Aufwand & Sehr großer Aufwand von nöten. Es muss alles selbst Entwickelt werden. & \label{krit_polyscope_aufwand}
	Bei kleinen Programmen kaum Aufwand. Aufwand steigt enorm bei mehr Anforderungen.\\ \hline
\end{tabular}
\captionof{table}{Zusammenfassung der Evaluierungskriterien für C-API und Polyscope}
\label{tab:vgl_interfaces_first}

\begin{tabular}{|l|p{0.33\textwidth}|p{0.33\textwidth}|}
	\hline
	\textbf{Kriterium} & \textbf{URScript} & \textbf{Eigener Adapter}\\ \hline \hline 
	Programmierung & verständnissvolle Dokumentation ermöglicht es einen schnellen Einstieg. Entwicklerwerkzeuge und \ac{Syntax Highlighting} erleichtern die Übersicht und Vereinfachen die Programmierung & Eine Etablierte Programmiersprache erleichtert das Programmieren deutlich. Vorhandene \acl{Librarys} erleichtern das Programmieren. \\
	\hline 
	Benutzerinteraktion & Die möglichkeiten bleiben wie bei Polyscope mit Popups beschränkt(siehe \ref{user_interaktion_polyscope_rel}) & Wie bei C-\ac{API} können \ac{GUI}'s erstellt werden, die komplexe Formulare und Menüs bieten. \\
	\hline 
	Testen & Mit gegebenen Mitteln sind automatische Tests nicht möglich. Es kann nur wie mit Polyscope Live getestet werden. & Automatische Test sind möglich. \\
	\hline
	Debuggen & Es gibt keine Möglichkeit zu Debuggen. URController liefert bei Fehler in der Syntax keine information & Fehler werden leicht gefunden, da die Etablierten Programmiersprachen den Programmcode nach Syntaxfehlern durchsuchen und anzeigen.
	\\
	\hline 
	Aufwand & Ähnlich wie bei Polyscope, jedoch etwas besser durch mehr Übersicht des Projektes & Anfangs ein Großer Aufwand von nöten. Bei mehreren Anwendungen für den Roboter ist Aufwand jedoch geringer als bei den anderen Schnittstellen.\\ 
	\hline 
\end{tabular}
\captionof{table}{Zusammenfassung der Evaluierungskriterien für URScript und Eigener Adapter}
\label{tab:vgl_interfaces_second}

\section{Vergleich der Schnittstellen}

Die C-API ist eine sehr Hardware nahe Schnittstelle zum Roboter. Der Aufwand der Betrieben werden muss ist sehr hoch. Diese Schnittstelle sollte nur in seltenen Fällen eingesetzt werden. Nur Spezielle Anwendungen die zur Laufzeit anpassungen an Bewegungssteuerung geben sollten diese Schnittstelle nutzen.
\\\\
Die Polyscope Software ist sehr gut geeignet für wenige komplexe Anwendungen, die keine bzw. kaum Benutzerinteraktion erfordern. Für eine Kollaboration die auf Interaktion angewiesen ist, ist diese Schnittstelle nicht zu empfehlen. Diese Schnittstelle kann nur sehr aufwändig auf persistente Daten zugreifen und speichern.
\\\\
URScript bietet in sachen Benutzerinteraktion nur die gleichen möglichkeiten, wie die Polyscope Software(siehe \ref{ur_script_user_interaction}). Es ist Übersichtlicher und verständlicher gegenüber Polyscope Anwendungen zu entwickeln, jedoch bietet die eigens entwickelte Scriptsprache nur wenig möglichkeiten, wirklich komplexe Anwendungen zu entwickeln.
\\\\
Der Eigene Adapter zum URController vereint die Vorteile einer etablierten Programmiersprache, nämlich vorhandene Entwicklerwerkzeuge und \ac{Software Bibliotheken} zu nutzen. Der Roboter wird über URScriptbefehle gesteuert wird, deshalb muss nicht tief in die Robotersteuerung eingreifen werden wie bei der C-API. Wenn viel Benutzerinteraktion von nöten ist, ist diese Schnittstelle zu empfehlen.

\section{Nicht erreichte Ziele}
\label{sec:Nicht_erreichte_ziele}

Mit der C-API konnte keine Anwendung geschrieben werden, die für die Evaluierungskriterien vollständige Daten liefern.