\chapter{Ergebnis}
\label{sec:ErreichteErgebnisse}

Im Folgenden Kapitel werden die Schnittstellen gegenübergestellt und verglichen. Desweiteren werden die nicht erreichten Ziele erörtert.
\\\\
\begin{tabular}{|l|p{0.333\textwidth}|p{0.333\textwidth}|}
	\hline
	\textbf{Kriterium} & \textbf{C-API} & \textbf{Polyscope} 
	\\ \hline \hline 
	Programmierbarkeit & Schon für einfache Roboterprogramme ist es schwer zu programmieren & Leichter Einstieg zum Programmieren für Anfänger
	\\ \hline
	Benutzerinteraktion &  Es ist möglich ein übersichtliches \& intuitives Interface zu entwickeln & 
	Keine komplexen Menüs möglich/ Nur schwache Interaktion möglich.
	\\ \hline
	Testen & Tests sind möglich, aber nur mit Simulation des Roboters & Keine eigenen Tests möglich/ Getestet wird immer Live an Roboter.
	\\ \hline
	Debuggen & Compiler findet Syntax Fehler \& Debugging ist nur mit simuliertem Roboter möglich & Beim Testen live am Roboter, bedingt möglich.
	\\ \hline
	Aufwand & Sehr großer Aufwand von nöten/ Es muss alles selbst entwickelt werden. & \label{krit_polyscope_aufwand}
	Bei kleinen Programmen kaum Aufwand/ Aufwand steigt enorm bei mehr Anforderungen.\\ \hline
\end{tabular}
\captionof{table}{Zusammenfassung der Evaluierungskriterien für C-API und Polyscope}
\label{tab:vgl_interfaces_first}

\begin{tabular}{|l|p{0.33\textwidth}|p{0.33\textwidth}|}
	\hline
	\textbf{Kriterium} & \textbf{URScript} & \textbf{Eigener Adapter}\\ \hline \hline 
	Programmierung & Verständliche Dokumentation ermöglicht einen schnellen Einstieg/ Entwicklerwerkzeuge und Syntax Highlighting erleichtern die Übersicht und vereinfachen die Programmierung & eine etablierte Programmiersprache und vorhandene Software Bibliotheken erleichtern das Programmieren. \\
	\hline 
	Benutzerinteraktion & Die Möglichkeiten bleiben wie bei Polyscope auf Popups beschränkt(siehe \ref{user_interaktion_polyscope_rel}) & Wie bei C-API können \ac{GUI}s erstellt werden, die komplexe Formulare und Menüs bieten \\
	\hline 
	Testen & Mit gegebenen Mitteln sind automatische Tests nicht möglich. Es kann nur wie mit Polyscope Live getestet werden. & Automatische Tests sind möglich \\
	\hline
	Debuggen & Keine Möglichkeit zu Debuggen/ URController liefert bei Fehler in der Syntax keine Information & Fehler werden leicht gefunden, da die etablierten Programmiersprachen den Programmcode nach Syntaxfehlern durchsuchen und anzeigen
	\\
	\hline 
	Aufwand & Ähnlich wie bei Polyscope, jedoch etwas besser durch mehr Übersicht des Projekts & Anfangs ein großer Aufwand von Nöten. Bei mehreren Anwendungen für den Roboter ist Aufwand jedoch geringer als bei den anderen Schnittstellen.\\
	\hline 
\end{tabular}
\captionof{table}{Zusammenfassung der Evaluierungskriterien für URScript und eigenem Adapter}
\label{tab:vgl_interfaces_second}

\section{Vergleich der Schnittstellen}

\textbf{Die C-API} ist eine sehr hardwarenahe Schnittstelle zum Roboter. Der Aufwand, der betrieben werden muss ist äußerst hoch. Diese Schnittstelle sollte nur in seltenen Fällen eingesetzt werden. Nur spezielle Anwendungen, die wärend der Laufzeit Anpassungen an die Bewegungssteuerung geben, sollten diese Schnittstelle nutzen.
\newpage
\textbf{Die Polyscope Software} eignet sich hervorragend für wenige komplexe Anwendungen, die keine bzw. kaum Benutzerinteraktion erfordern. Für eine Kollaboration die auf direkte Kommunikation zwischen Mensch und Roboter angewiesen ist, ist diese Schnittstelle nicht zu empfehlen, denn diese sie kann nur sehr aufwändig auf persistente Daten zugreifen und speichern.
\\\\
\textbf{URScript} bietet in Sachen Benutzerinteraktion nur die gleichen Möglichkeiten wie die Polyscope Software(siehe \ref{ur_script_user_interaction}). Es ist übersichtlicher und verständlicher gegenüber Polyscope-Anwendungen zu entwickeln, bietet jedoch die eigens entwickelte Scriptsprache nur wenig Spielraum, um wirklich komplexe Anwendungen zu entwickeln.
\\\\
\textbf{Ein eigener Adapter} zum URController vereint die Vorteile einer etablierten Programmiersprache, nämlich vorhandene Entwicklerwerkzeuge und Software-Bibliotheken zu nutzen. Der Roboter wird über URScriptbefehle gesteuert, deshalb muss nicht tief in die Robotersteuerung, wie bei der C-API eingreifen werden. Wenn viel Benutzerinteraktion notwendig ist, ist diese Schnittstelle empfehlenswert.

\section{Nicht erreichte Ziele}
\label{sec:Nicht_erreichte_ziele}

Die aufgetretenen Probleme mit der C-API haben verhindert, dass keine vollständige Anwendung geschrieben werden konnte. Aus Zeitmangel wurde dort die Entwicklung abgebrochen. 
Die Fehler konnten ein wenig eingegrenzt werden. Ein Hardwaredefekt ist nicht auszuschließen. 
\\
Es ist nicht bekannt, welche C-API Version in dem URController benutzt wird. Die selbst erstellten Bewegungsprofile sehen korrekt aus, jedoch konnte man sehen, dass der URController funktionierende Soll-Werte übergeben hat. Es ist möglich, dass dem URController mehr Funktionen in der C-API zur Verfügung stehen.