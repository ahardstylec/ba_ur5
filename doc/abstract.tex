\section*{Abstract}
\label{abstract}

Um die Vorteile des kollaborativen Arbeitens von Menschen und Robotern anzuwenden, wird im Zuge dieser Arbeit der für die Kollaboration zugelassene Roboter UR5 der Firma ``Universal Robots'' untersucht. Es wird Untersucht, welche Möglichkeiten diesen Roboter zu programmieren möglich sind. Die Untersuchung erfolgt aufgrund einiger Kriterien, die auf den Einsatz mit kollaboration zielen.
Die Schnittstellen des UR5 Roboters werden untersucht und dokumentiert.
Am Ende dieser Arbeit wird eine Entscheidungsfindung zusammengefasst, welche Schnittstelle zu welchem Anwendungsfall am besten zu wählen ist. Um dies zu evaluieren, wurde eine Beispielanwendung in jeder Schnittstelle entwickelt. Die Ergebnisse sind am Ende knapp zusammengefasst.