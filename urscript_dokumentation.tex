\documentclass[
a4paper,
12pt,
oneside,
headsepline,		% Linie für Kopfzeile
footsepline,		% Linie für Fußzeile
%bibtotoc
]{scrbook}
 
% Druckbereich: \areaset[BCOR]{textwidth}{textheight}
% BCOR ist "Binding Correction", also wieviel Innenrand verloren geht
% A4 hat 297mm x 210mm
% wenn keine Marginalien, dann ist Breite 15cm vielleicht besser
\areaset[1.5cm]{14cm}{25cm}
 
%% Die folgende Zeile sorgt dafür, daß die Fußnoten eingerückt werden,
%% und zwar um 2em (class scrbook).
\deffootnote{2em}{2em}{\textsuperscript{\normalfont\thefootnotemark} }
 
\usepackage[T1]{fontenc}      % Unterstützung für Europäische-Zeichen-Ausgabe
\usepackage{ae}               % verbesserte Unterstützung für Umlaute
\usepackage[german]{babel}    % deutsche Übersetzungen und Wortumbrüche
\usepackage[scaled]{helvet}  % schönere Schriftart: Helvetica
\usepackage{mathptmx}            % passende Mathe-Schriftart
\usepackage{courier}             % passende Monospaced-Schriftart
\usepackage{pgf}              % Unterstützung für Graphiken
\usepackage{tikz}             % Unterstützung für Graphiken
\usepackage[onehalfspacing]{setspace}
\usepackage{acronym} 
\usepackage{listings}
\usepackage[utf8x]{inputenc}  % Unterstützung für Unicode-Zeichen-Eingabe
\usepackage{color}
\definecolor{Gray}{gray}{0.9}
\definecolor{sun1}{rgb}{0.2,0.2,0.4}
\definecolor{sun2}{rgb}{0.4,0.4,0.6}
\definecolor{sun3}{rgb}{0.6,0.6,0.8}
\definecolor{sun4}{rgb}{0.8,0.8,1}
\definecolor{msblau}{rgb}{0.31,0.4,0.517}
\definecolor{darkred}{rgb}{0.5,0,0}
\definecolor{darkgreen}{rgb}{0,0.5,0}
\definecolor{darkblue}{rgb}{0,0,0.5}
 
\usepackage[                
   pdftex,                  % Ausgabe-Medium: PDF
   colorlinks=true,         % farbige Links in der Bildschirm-Version?
   pdfstartview=FitV,       % wie soll Acrobat starten?
   linkcolor=blue,         % Farbe für Querverweise
   citecolor=blue,         % Farbe für Zitierungen
   urlcolor=blue,          % Farbe für Links
   bookmarks=true
   ]{hyperref}              % Paket für Links im PDF
 
%%%% Informationen über den Text festlegen %%%%%%%%%%%%%%%%%%%%%%%%%%%%%%%%%%
\title{Dokumenation der C-API mit Beispielanwendung}
\author{Andreas Collmann}
\date{\today}
 
%%% hier können noch viel viel mehr Einstellungen kommen
%%%% hier beginnt der Inhalt %%%%%%%%%%%%%%%%%%%%%%%%%%%%%%%%%%%%%%%%%%%%%%%%
%\spacing{1.5}

\makeindex

\newcommand{\eruck}[1]{
    \noindent\hspace*{#1}
}
\newcommand{\zab}[1]{
    \vspace*{#1}
}
\newcommand{\fina}[1]{
    {\em #1}
}
\newcommand{\finar}[1]{
    {\fina{#1}\textsuperscript{\textregistered}}		
}
\newcommand{\sona}[1]{
    {\em #1}
}
\newcommand{\code}[1]{
    {\\ \eruck{20mm} \texttt{#1} \\}
}

% Formatierung Quellcode

\lstset{
    language=Python,
	numbers=left,
    breaklines=true,
	backgroundcolor=\color{Gray},
	basicstyle=\small,
	showspaces=false,
    basicstyle=\scriptsize\ttfamily,
    keywordstyle=\bfseries\ttfamily\color{orange},
    stringstyle=\color{blue}\ttfamily,
    commentstyle=\color{gray}\ttfamily,
    emph={square}, 
    emphstyle=\color{blue}\texttt,
    emph={[2]root,base},
    emphstyle={[2]\color{orange}\texttt},
    showstringspaces=false,
    flexiblecolumns=false,
    tabsize=2,
    numbers=left,
    numberstyle=\tiny,
    numberblanklines=false,
    stepnumber=1,
    numbersep=10pt,
    xleftmargin=15pt
	}


\begin{document}

 \tableofcontents

\chapter{URscript Beispielanwendung}

\begin{lstlisting}
def myProg():

  global richtung=False

  // festlegen der maximalen beschleunigung über die a_max variable
  global a_max = d2r(40)

  // festlegen er maximalen geschwindigkeit über die v_max variable
  global v_max = d2r(60)

  // Startpunkt für blaue seite des würfels
  global blue_deg = [d2r(79.34), d2r(-130.29), d2r(-108.28), d2r(-31.43), d2r(-90.0), d2r(-1.02)]


  // Funktione rotiert den Würfel über die übergebene Aches (1= X-Achse, 2=Y-Achse, 3=Z-Achse) um a grad
  def rotate_axis(a, b):
    rotationPos=p[0.0, 0.0, 0.0, 0.0, 0.0, 0.0]
    a= d2r(a)
    if b == 1:
      rotationPos = p[0.0, 0.0, 0.0, a, 0.0, 0.0]
    elif b == 2:
      rotationPos = p[0.0, 0.0, 0.0, 0.0, a, 0.0]
    elif b == 3:
      rotationPos = p[0.0, 0.0, 0.0, 0.0, 0.0, a]
    end
    movel(pose_add(get_actual_tcp_pose(), rotationPos), a=0.5, v=0.1)
  end


  // Funktion dreht den letzten Joint bei digital in HIGH zur Nächsten form um 71.299 grad, Bis alle formen auf der Ebene erreicht wurden
  def move_toy():
    schleife_1 = 4
    while schleife_1 > 0:
      while digital_in[1] == False:
        sync()
      end
      r = d2r(71.299)
      pos = get_actual_joint_positions()
      pos= [pos[0], pos[1], pos[2], pos[3], pos[4], pos[5]+r]
      movej(pos, a=a_max, v=v_max)
      schleife_1=schleife_1-1
    end
  end

  // justiert den roboter in die richtung vorgegeben von der variable direction
  thread justify():
    if direction  ==  True:
      move_z=0.5
    else:
      move_z=-0.5
    end
    movel(pose_add(get_actual_tcp_pose(), p[0.0,0.0, move_z, 0.0,0.0,0.0]), a=0.1, v=0.15)
  end

  // Funktion Fragt Spieler ob der Robotre in der höhe Justiert werden soll
  def justify_height():
    // Ja/Nein anfrage vom Benutzer holen ob justiert wird
    global justieren = request_boolean_from_primary_client("roboter justieren?")
    varmsg("justieren", justieren)
    while (justieren ==  True ):
    // Ja/Nein anfrage vom Benutzer holen in welche richtung justiert wird
      global richtung = request_boolean_from_primary_client("hoch ~> ja ; runter ~> nein")
      varmsg("richtung", richtung)
      # wait until button 1 is pressed
      while digital_in[1] == False:
        sync()
      end
      // justify function bewegt den Roboter solange der digitale eingang auf high steht in die vorgegebene richtung
      justify_handler= run justify()
      while digital_in[1] == True:
        sync()
      end
      kill justify_handler
      blue_deg= get_actual_joint_positions()
      justieren = request_boolean_from_primary_client("roboter erneut justieren?")
      varmsg("justieren", justieren)
    end
  end

  // funktion fragt ob ein neuer spieler erstellt werden soll, wenn ja, wird über die socket verbindung der spieler mit dem neuen Namen gespeichert.
  def spieler_erst():
    neu = request_boolean_from_primary_client("Neuen Spieler erstellen?")
    name = request_string_from_primary_client("Name des Spielers?")
    if neu == True:
      socket_send_string("new_patient")
      socket_send_string(name)
    end
  end

  // Abfrage vom programm bei programmstart ob das programm weiterlaufen soll.
  start_var = request_boolean_from_primary_client("program starten?")
  varmsg("start_var", start_var)

  // öffnen der socket verbindung zum Daten Server
  data_server = socket_connect("141.100.101.48", 8000)
  
  
  if start_var == True:
    movej(blue_deg, a=a_max, v=v_max)
    if( data_server):
      spieler_erst()
    else:
      global p_name = request_string_from_primary_client("name des Spielers?")
      varmsg("p_name", p_name)
    end
    justify_height()

    // laufe solange gespielt werden soll
    while start_var == True:
      movej(blue_deg, a=a_max, v=v_max)
      popup("Wenn Wuerfel entleet ist, kann losgelegt werden(Mit Button 1 den Wuerfel drehen.)", "Nachricht", False, False)
      move_toy()
      movej(red_deg, a=a_max, v=v_max)
      rotate_axis(-40, 1)
      move_toy()
      start_var = request_boolean_from_primary_client("erneut beginnen?")
      varmsg("start_var", start_var)
    end
  end
  popup("programm wird beendet", "Nachricht", False, False)
  socket_close()
end
\end{lstlisting}

\chapter{Programm zum laden des URScripts auf den URController - Ruby}

Datei ur5.rb beinhält die Classen zum verbinden zum Server und senden von Script
\begin{lstlisting}
require 'thread'
require './deserialize.rb'

def recv_interface(interface)
  while true
    buf = interface.socket.recv(1254)
    RobotState.unpack(buf, interface.robot_state)
  end
end

module Ur5
  class Script
    attr_accessor :script, :scriptname

    def initialize(scriptname, debug=false)
      @scriptname = scriptname
      read_script
      @debug=debug
    end

    def read_script
      @script = ""
      puts("loading script..") if @debug
      script_file_socket = File.open(@scriptname, 'r')
      while(line = script_file_socket.gets)
        unless line =~ /#/
          @script << line
          puts line if @debug
        end
      end
      @script
    end

    def manipulate(text, val)
      puts "manipulate" if @debug
      abort("cant find text #{text} in script, cant manipulate") unless @script.gsub!(/#{text}/, val)
      puts @script if @debug
      @script
    end
  end

  class Controller
    attr_accessor :script, :secondary_interface, :primary_interface, :ip_address, :primary_port, :seconary_port

    def initialize(ip_address, ports={}, debug=false)
      self.ip_address= ip_address
      self.primary_port= ports[:primary] || 30001
      self.seconary_port= ports[:seconary] || 30002
      @debug=debug
    end

    def connect_controller
      puts("connect to Ur5Controller(#{self.ip_address})") if @debug
      begin
        self.secondary_interface= SecondaryInterface.new(self.ip_address, self.seconary_port) unless self.secondary_interface && self.secondary_interface.connected?
        # self.primary_interface= PrimaryInterface.new(self.ip_address, self.primary_port) unless self.primary_interface && self.primary_interface.connected?
        self.secondary_interface.connect
        # self.primary_interface.connect
      rescue Exception => e
        abort ("cant connect to Ur5SecondaryInterface: #{e.message}")
      end
    end

    def disconnect_controller
      self.secondary_interface.disconnect if self.secondary_interface
      self.primary_interface.disconnect if self.primary_interface
      puts("disconnect from Ur5Controller") if @debug
    end

    def send_script(ur5_script)
      puts("send script") if @debug
      puts ur5_script.script if @debug
      self.secondary_interface.send(ur5_script.script)
    end

    def test_controller
      begin
        @ur5.connect_controller unless @ur5.connected?
        script = Ur5::Script.new
        script.script="set digital out(0, True)"
        # TODO parse secondary interface for masterData and DigitalOut bits == 1
      rescue
        abort("connection to robot not stable")
      end
    end
  end

  
  class Ur5Interface
    attr_accessor :socket, :ip_address, :port, :robot_state
    
    def initialize(ip_address, port)
      self.ip_address= ip_address
      self.port= port
    end

    def connect
      self.socket= TCPSocket.new(self.ip_address, self.port) unless self.socket
      Thread.new{recv_interface(self)}
    end

    def connected?
      self.socket ? !self.socket.closed? : nil
    end
    
    def disconnect
      self.socket.close if self.connected?
      self.socket=nil
    end
  end

  class SecondaryInterface < Ur5Interface
    def initialize(*args)
      super(*args)
    end

    def send(txt)
      self.socket.puts(txt) if self.connected?
    end
  end
end
\end{lstlisting}

load\_programs.rb. Hauptprogram nutzt ur5.rb lib um script an controller zu senden

\begin{lstlisting}
#!/home/andi/.rvm/rubies/ruby-1.9.3-p194/bin/ruby

require 'socket'
require './ur5.rb'

debug=false
if ARGV.length > 1 && ARGV.last == "debug"
  debug=true
  ARGV.pop
end

@ur5 = Ur5::Controller.new("141.100.101.20", {}, debug)

at_exit do
  @ur5.disconnect_controller if @ur5
  puts "quit(#{($!.nil? && !$!.respond_to?(:status))? 0 : $!.status}) and disconnect" if debug
end

# @ur5.test_controller
@ur5.connect_controller

abort("no program given") if ARGV.empty?
abort("program #{ARGV[0]} not found") unless File.exists?(ARGV[0])

script = Ur5::Script.new(ARGV[0], debug)
  
#----- manipulate script here

script.manipulate("<%angle%>", ARGV[1]) if(ARGV[1])

#-----

# @ur5.send_script(script)

while true
end
\end{lstlisting}

deserialize.rb. Module zum deserialisieren der Datenpackete von der Secondary Schnittstelle

\begin{lstlisting}
module PackageType
    ROBOT_MODE_DATA = 0
    JOINT_DATA = 1
    TOOL_DATA = 2
    MASTERBOARD_DATA = 3
    CARTESIAN_INFO = 4
    KINEMATICS_INFO = 5
    CONFIGURATION_DATA = 6
    FORCE_MODE_DATA = 7
    ADDITIONAL_INFO = 8
end


module RobotMode
    RUNNING = 0
    FREEDRIVE = 1
    READY = 2
    INITIALIZING = 3
    SECURITY_STOPPED = 4
    EMERGENCY_STOPPED = 5
    FATAL_ERROR = 6
    NO_POWER = 7
    NOT_CONNECTED = 8
    SHUTDOWN = 9
    SAFEGUARD_STOP = 10
end

class RobotModeData
  attr_accessor 'timestamp', 'robot_connected', 'real_robot_enabled',
                 'power_on_robot', 'emergency_stopped',
                 'security_stopped', 'program_running', 'program_paused',
                 'robot_mode', 'speed_fraction'

    def self.unpack(buf)
        rmd = RobotModeData.new
        rmd.timestamp, rmd.robot_connected, 
        rmd.real_robot_enabled, rmd.power_on_robot,
        rmd.emergency_stopped, rmd.security_stopped, rmd.program_running,
        rmd.program_paused, rmd.robot_mode, rmd.speed_fraction= buf.unpack("Q>CCCCCCCCG")
    rmd.robot_connected = rmd.robot_connected == 1
    rmd.real_robot_enabled = rmd.real_robot_enabled == 1
    rmd.power_on_robot = rmd.power_on_robot == 1
    rmd.emergency_stopped = rmd.emergency_stopped == 1
    rmd.security_stopped = rmd.security_stopped == 1
    rmd.program_running = rmd.program_running == 1
    rmd.program_paused = rmd.program_paused == 1
    end
end

class JointData
    attr_accessor 'q_actual', 'q_target', 'qd_actual',
                 'i_actual', 'v_actual', 't_motor', 't_micro', 'joint_mode'
    def self.unpack(buf)
        all_joints = []
        offset = 0
        (0..5).each do |i|
            jd = JointData.new()
            jd.q_actual, jd.q_target, jd.qd_actual, jd.i_actual, jd.v_actual, jd.t_motor, jd.t_micro, jd.joint_mode = buf[offset..-1].unpack("GGGggggC")
            offset += 41
            all_joints<< jd
        end
        return all_joints
    end
end


class ToolData
    attr_accessor 'analog_input_range2', 'analog_input_range3',
                 'analog_input2', 'analog_input3',
                 'tool_voltage_48V', 'tool_output_voltage', 'tool_current',
                 'tool_temperature', 'tool_mode'
    
    def self.unpack(buf)
        td = ToolData.new
        td.analog_input_range2, td.analog_input_range3,
        td.analog_input2, td.analog_input3,
        td.tool_voltage_48V, td.tool_output_voltage, td.tool_current,
        td.tool_temperature, td.tool_mode = buf.unpack("ccGGgCggC")
        return td
  end
end

class MasterboardData
    attr_accessor 'digital_input_bits', 'digital_output_bits',
                 'analog_input_range0', 'analog_input_range1',
                 'analog_input0', 'analog_input1',
                 'analog_output_domain0', 'analog_output_domain1',
                 'analog_output0', 'analog_output1',
                 'masterboard_temperature',
                 'robot_voltage_48V', 'robot_current',
                 'master_io_current', 'master_safety_state',
                 'master_onoff_state'
    
    def self.unpack(buf)
        md = MasterboardData.new
    md.digital_input_bits, md.digital_output_bits,
    md.analog_input_range0, md.analog_input_range1,
    md.analog_input0, md.analog_input1,
    md.analog_output_domain0, md.analog_output_domain1,
    md.analog_output0, md.analog_output1,
    md.masterboard_temperature,
    md.robot_voltage_48V, md.robot_current,
    md.master_io_current, md.master_safety_state,
    md.master_onoff_state = buf.unpack("s!>s!>ccGGccGGggggCCc")
        return md
  end 
end

class CartesianInfo
    attr_accessor 'x', 'y', 'z', 'rx', 'ry', 'rz'
    def self.unpack(buf)
        ci = CartesianInfo.new
      ci.x, ci.y, ci.z, ci.rx, ci.ry, ci.rz = buf.unpack("GGGGGG")
        return ci
    end
end

class ForceModeData
    attr_accessor 'x', 'y', 'z', 'rx', 'ry', 'rz', 'robot_dexterity'
    
    def self.unpack(buf)
        fmd = ForceModeData.new
        fmd.x, fmd.y, fmd.z, fmd.rx, fmd.ry, fmd.rz, fmd.robot_dexterity = buf.unpack("GGGGGGG")
        return fmd
    end
end

class AdditionalInfo
    attr_accessor 'ctrl_bits', 'teach_button'
    
    def self.unpack(buf)
        ai = AdditionalInfo.new
        ai.ctrl_bits, ai.teach_button = buf.unpack("CC")
    ai.teach_button= ai.teach_button == 1
        return ai
    end
end

class JointLimitData
    attr_accessor 'min_limit', 'max_limit', 'max_speed', 'max_acceleration'
end

class ConfigurationData
    attr_accessor 'joint_limit_data',
                 'v_joint_default', 'a_joint_default', 
                 'v_tool_default', 'a_tool_default', 'eq_radius',
                 'dh_a', 'dh_d', 'dh_alpha', 'dh_theta',
                 'masterboard_version', 'controller_box_type',
                 'robot_type', 'robot_subtype'
    def self.unpack(buf)
        cd = ConfigurationData.new
        cd.joint_limit_data = []
        begin
          (0..6).to_a.each do |i|
              jld = JointLimitData.new
              jld.min_limit, jld.max_limit = buf[(5+16*i)..-1].unpack("GG")
              jld.max_speed, jld.max_acceleration = buf[(5+16*6+16*i)..-1].unpack("GG")
              cd.joint_limit_data.push(jld)
          end
          cd.v_joint_default, cd.a_joint_default, cd.v_tool_default, cd.a_tool_default,cd.eq_radius = buf[5+32*6..-1].unpack("GGGGG")
          cd.masterboard_version, cd.controller_box_type, cd.robot_type, cd.robot_subtype = buf[5+32*6+5*8+6*32..-1].unpack("L>l>l>l>")
      rescue Exception => e
        puts e.message
      end
      return cd
    end
end

class KinematicsInfo
    def self.unpack(buf)
        KinematicsInfo.new
    end
end

class RobotState
  include PackageType
  attr_accessor 'robot_mode_data', 'joint_data', 'tool_data',
                 'masterboard_data', 'cartesian_info',
                 'kinematics_info', 'configuration_data',
                 'force_mode_data', 'additional_info',
                 'unknown_ptypes'
    def initialize()
      self.unknown_ptypes=[]
    end

    def self.unpack(buf, rs)
        return if(buf.size<5)
        length, mtype = buf[0..4].unpack("L>C")
    if length < len(buf)
            print("got to many packages")
            return RobotState.unpack(buf[buf.size-length..-1])
        elsif length > len(buf)
            abort("Could not unpack packet: length field is incorrect")
        end
        if mtype != 16
            if mtype == 20
                print("Likely a syntax error:")
                print(buf)
            end
        end
        rs = self.new unless rs
        offset = 5
        while offset < buf.size do
          length, ptype = buf[offset..-1].unpack("L>C")
          offset+=5
            package_buf = buf[offset..offset+length-5]
          puts("unpacking package #{ptype.ord}")
            case ptype
              when PackageType::ROBOT_MODE_DATA
                rs.robot_mode_data = RobotModeData.unpack(package_buf)
              when PackageType::JOINT_DATA
                rs.joint_data = JointData.unpack(package_buf)
              when PackageType::TOOL_DATA
                rs.tool_data = ToolData.unpack(package_buf)
              when PackageType::MASTERBOARD_DATA
                rs.masterboard_data = MasterboardData.unpack(package_buf)
              when PackageType::CARTESIAN_INFO
                rs.cartesian_info = CartesianInfo.unpack(package_buf)
              when PackageType::FORCE_MODE_DATA
                rs.force_mode_data = ForceModeData.unpack(package_buf)
              when PackageType::ADDITIONAL_INFO
                rs.additional_info = AdditionalInfo.unpack(package_buf)
              when PackageType::CONFIGURATION_DATA
                rs.configuration_data = ConfigurationData.unpack(package_buf)
              when PackageType::KINEMATICS_INFO
                rs.kinematics_info = KinematicsInfo.unpack(package_buf)
              else
                rs.unknown_ptypes << ptype
            end
            offset += length-5
        end
    end
end
\end{lstlisting}

\chapter{Programm zum laden des URScripts auf den URController - Python}

\begin{lstlisting}
\end{lstlisting}

\end{document}